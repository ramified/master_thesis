%%%%%%%%%%%%%%%%%%%%%%%%%%%%%%%%%%%%%%%%%%%%%%%%%%%%%%%%%%%%%%%%%%%%%%%%%%%%%%%%%%%%%%%%%%%%%
\chapter{Excess intersection formula}\label{chap:excess_intersection_formula}
Finally, we are able to compute the convolution structure of the Steinberg variety in this Chapter. We first introduce the convolution product, then give an explicit intersection formula, and finally apply theorems to our settings.

\section{Convolution}\label{sec:convolution}
The construction of the convolution product has a similar flavor with Fourier-Mukai transformation, which is the composition of pullback, tensor product and proper pushforward.

\begin{defn}[Convolution product]\label{def:convolution_product}
For the convenience of reading, we divide the whole process into three steps.
\paragraph*{\underline{Step1.}}Setting.\\[-3mm]

Let $M_1$, $M_2$, $M_3$ be smooth quasi-projective $G$-varieties. For convenience, denote
$$M_{ij}:=M_i \times M_j \qquad M_{123}=M_1 \times M_2 \times M_3$$
and $p_i^{jk}, p_i:=p_i^{123}, p_{ij}:=p_{ij}^{123}$ as projections onto some factors, as follows.
% https://q.uiver.app/?q=WzAsNyxbMiwwLCJHIFxcdGltZXMgRyBcXHRpbWVzIFgiXSxbMSwxLCJHIFxcdGltZXMgRyJdLFszLDEsIkcgXFx0aW1lcyBYIl0sWzUsMSwiRyBcXHRpbWVzIFgiXSxbNCwyLCJYIl0sWzIsMiwiRyJdLFswLDIsIkciXSxbMSw2LCJwXzFeezEyfSIsMl0sWzAsMSwicF97MTJ9IiwyXSxbMCwyLCJwX3syM30iXSxbMiw0LCJwXzNeezIzfSJdLFsxLDUsInBfMl57MTJ9Il0sWzIsNSwicF8yXnsyM30iLDJdLFszLDQsInBfM157MTN9IiwwLHsibGFiZWxfcG9zaXRpb24iOjMwLCJzdHlsZSI6eyJib2R5Ijp7Im5hbWUiOiJkYXNoZWQifX19XSxbMyw2LCJwXzFeezEzfSIsMix7ImxhYmVsX3Bvc2l0aW9uIjoyMCwic3R5bGUiOnsiYm9keSI6eyJuYW1lIjoiZGFzaGVkIn19fV0sWzAsMywicF97MTN9IiwwLHsiY3VydmUiOi0yfV1d
\[\begin{tikzcd}[column sep={1.5cm,between origins}]
	&& {M_{123}} &&&[2cm]\\
	& {M_{12}} && {M_{23}} && {M_{13}} \\
	M_1 && M_2 && M_3 &
	\arrow["{p_1^{12}}"', from=2-2, to=3-1]
	\arrow["{p_{12}}"', from=1-3, to=2-2]
	\arrow["{p_{23}}", from=1-3, to=2-4]
	\arrow["{p_3^{23}}", from=2-4, to=3-5]
	\arrow["{p_2^{12}}", from=2-2, to=3-3]
	\arrow["{p_2^{23}}"', from=2-4, to=3-3]
	\arrow["{p_3^{13}}"{pos=0.3}, dashed, from=2-6, to=3-5]
	\arrow["{p_1^{13}}"'{pos=0.2}, dashed, from=2-6, to=3-1]
	\arrow["{p_{13}}", curve={height=-12pt}, from=1-3, to=2-6]
\end{tikzcd}\]
(Check that $p_i = p_i^{jk} \circ p_{jk}$ for $1 \leqslant j < k \leqslant 3$, $i=j$ or $i=k$)



\paragraph*{\underline{Step2.}}Convolution product on the level of varieties.\\[-3mm]

For closed $G$-subvarieties $Z_{12} \subseteq M_{12}$, $Z_{23} \subseteq M_{23}$, denote
$$Z_{123}:= p_{12}^{-1}(Z_{12}) \cap p_{23}^{-1}(Z_{23}) \subseteq M_{123}$$
as the intersection of two preimages. The \textbf{convolution product} of $Z_{12}$ and $Z_{23}$ is defined as
$$Z_{12} \circ Z_{23} := p_{13}(Z_{123}) \subseteq M_{13}$$
which is a closed $G$-subvariety of $M_{13}$.
\paragraph*{\underline{Step3.}}Convolution product on the level of $K$-theories.\\[-3mm]

Denote 
$$\pi_{12}:=p_{12}\big|_{p_{12}^{-1}(Z_{12})} \qquad \pi_{23}:=p_{23}\big|_{p_{23}^{-1}(Z_{23})} \qquad
\pi_{13}:=p_{13}\big|_{Z_{123}}$$
as corresponding morphisms restricted to $p_{12}^{-1}(Z_{12})$, $p_{23}^{-1}(Z_{23})$ and  $Z_{123}$, respectively. We assume that $\pi_{13}$ is proper, so that we can use proper pushforward in $K$-theory.

We define the convolution product by
$$*: K_0^G(Z_{12}) \times K_0^G(Z_{23}) \longrightarrow K_0^G(Z_{12} \circ Z_{23}) \qquad (\mathcal{F}_{12}, \mathcal{F}_{23}) \longmapsto \mathcal{F}_{12}*\mathcal{F}_{23}$$
$$\mathcal{F}_{12}*\mathcal{F}_{23}=\pi_{13,*}\left(\pi_{12}^* \mathcal{F}_{12} \otimes \pi_{23}^* \mathcal{F}_{23} \right) \in K_0^G(Z_{12} \circ Z_{23})$$

\end{defn}

\begin{remark}
Those "$Z$-varieties" ($Z_{12}$, $p_{12}^{-1}(Z_{12})$, $Z_{123}$, etc.) are often singular in practice, so $\pi_{12}^{*}$, $\pi_{23}^{*}$ and $\otimes$ are defined in the sense of "restriction with supports", under the "$M$-varieties" which are smooth. The following diagram best illustrates the "actual" definition.
%closed immersion induce f.faithful map in quasicoherent sheaf: https://en.wikipedia.org/wiki/Closed_immersion
% https://q.uiver.app/?q=WzAsOCxbMCwwLCJLXzBeRyhaX3sxMn0pIFxcdGltZXMgS18wXkcoWl97MjN9KSJdLFsxLDAsIktfMF5HXFxiaWcocF97MTJ9XnstMX0oWl97MTJ9KVxcYmlnKSBcXHRpbWVzIEtfMF5HXFxiaWcocF97MTJ9XnstMX0oWl97MjN9KVxcYmlnKSJdLFswLDEsIktfMF5HKE1fezEyfSkgXFx0aW1lcyBLXzBeRyhNX3syM30pIl0sWzEsMSwiS18wXkcoTV97MTIzfSkgXFx0aW1lcyBLXzBeRyhNX3sxMjN9KSJdLFsyLDEsIktfMF5HKE1fezEyM30pIl0sWzMsMSwiS18wXkcoTV97MTN9KSJdLFsyLDAsIktfMF5HKFpfezEyM30pIl0sWzMsMCwiS18wXkcoWl97MTJ9IFxcY2lyYyBaX3syM30pIl0sWzAsMiwiIiwwLHsic3R5bGUiOnsidGFpbCI6eyJuYW1lIjoiaG9vayIsInNpZGUiOiJib3R0b20ifX19XSxbMSwzLCIiLDAseyJzdHlsZSI6eyJ0YWlsIjp7Im5hbWUiOiJob29rIiwic2lkZSI6ImJvdHRvbSJ9fX1dLFs2LDQsIiIsMCx7InN0eWxlIjp7InRhaWwiOnsibmFtZSI6Imhvb2siLCJzaWRlIjoiYm90dG9tIn19fV0sWzcsNSwiIiwwLHsic3R5bGUiOnsidGFpbCI6eyJuYW1lIjoiaG9vayIsInNpZGUiOiJib3R0b20ifX19XSxbMiwzLCJwX3sxMn1eKiBcXHRpbWVzIHBfezIzfV4qIl0sWzMsNCwiXFxvdGltZXMiXSxbNCw1LCJwX3sxMywqfSJdLFs2LDcsIlxccGlfezEzLCp9Il0sWzEsNiwiXFxvdGltZXMiLDAseyJzdHlsZSI6eyJib2R5Ijp7Im5hbWUiOiJkYXNoZWQifX19XSxbMCwxLCJcXHBpX3sxMn1eKiBcXHRpbWVzIFxccGlfezIzfV4qIiwwLHsic3R5bGUiOnsiYm9keSI6eyJuYW1lIjoiZGFzaGVkIn19fV1d
\tikzcdset{scale cd/.style={every label/.append style={scale=#1},
    cells={nodes={scale=#1}}}}
\begin{equation}\label{eq:convolution_singularity_solved}
\begin{tikzcd}[scale cd=0.7]
	{K_0^G(Z_{12}) \times K_0^G(Z_{23})} & {K_0^G\big(p_{12}^{-1}(Z_{12})\big) \times K_0^G\big(p_{12}^{-1}(Z_{23})\big)} & {K_0^G(Z_{123})} & {K_0^G(Z_{12} \circ Z_{23})} \\
	{K_0^G(M_{12}) \times K_0^G(M_{23})} & {K_0^G(M_{123}) \times K_0^G(M_{123})} & {K_0^G(M_{123})} & {K_0^G(M_{13})}
	\arrow["{\iota_{Z_{12},*}, \iota_{Z_{23},*}}",from=1-1, to=2-1]
	\arrow[from=1-2, to=2-2]
	\arrow[from=1-3, to=2-3]
	\arrow["{\iota_{Z_{12} \circ Z_{23},*}}"', from=1-4, to=2-4]
	\arrow["{p_{12}^* \times p_{23}^*}", from=2-1, to=2-2]
	\arrow["\otimes", from=2-2, to=2-3]
	\arrow["{p_{13,*}}", from=2-3, to=2-4]
	\arrow["{\pi_{13,*}}", from=1-3, to=1-4]
	\arrow["\otimes", dashed, from=1-2, to=1-3]
	\arrow["{\pi_{12}^* \times \pi_{23}^*}", dashed, from=1-1, to=1-2]
\end{tikzcd}
\end{equation}
Somewhat lucky, the diagram in \eqref{eq:convolution_singularity_solved} commutes by the vanishment of the Euler class. Therefore, one can compute
$$\mathcal{F}_{12}*\mathcal{F}_{23}=p_{13,*}\left(p_{12}^* \iota_{Z_{12},*}\mathcal{F}_{12} \otimes p_{23}^* \iota_{Z_{23},*}\mathcal{F}_{23} \right) \in K_0^G(M_{13}),$$
and then find the preimage of it under the map $\iota_{Z_{12} \circ Z_{23},*}$. This technique will be used in Subsection \ref{subsec:convolution_product_fml}.
\end{remark}

The whole process can be concluded in Figure \ref{fig:convolution_product}.
\begin{figure}[ht]
  \vspace{0cm}
    \centering  
    \includegraphics[]{figures/table/figure_convolution.pdf}    
    \caption{Convolution Product}  
    \label{fig:convolution_product}
\end{figure}

\section{Statement}
To facilitate the computation of intersection (i.e., tensor product in the construction of convolution product), we state the excess intersection formula.

\begin{theorem}[{Excess intersection formula, \cite[Corollary 9.4]{przezdziecki2015geometric}}]\label{thm:excess_intersection_formula}
Let $X'$ be a smooth $G$-variety, $X \subseteq X'$ be a (maybe singular) closed $G$-subvariety, and $Y_1,Y_2 \subseteq X$ be closed $G$-equivariant embeddings (of globally finite $\Tor$-dimension). Denote
\begin{equation*}
\begin{aligned}
  Y:=\;&Y_1 \cap Y_2 \qquad \iota_Y: Y \hookrightarrow X  \\
  \mathcal{T}:=\;&TX\big|_{Y} / \left( TY_1\big|_{Y}+ TY_2\big|_{Y}   \right) 
\end{aligned}
\end{equation*}
\begin{equation}\label{eq:excess_intersection_formula}
% https://q.uiver.app/?q=WzAsNyxbMSwxLCJXIl0sWzMsMSwiWiJdLFszLDIsIlgiXSxbMSwyLCJZIl0sWzQsMCwiXFxtYXRoY2Fse059X3tcXHZhcnBoaX0iXSxbMiwwLCJnXnsqfVxcbWF0aGNhbHtOfV97XFx2YXJwaGl9Il0sWzAsMCwiXFxtYXRoY2Fse059X3tcXHBoaX0iXSxbMywyLCJmIl0sWzEsMiwiXFx2YXJwaGkiXSxbMCwxLCJnIl0sWzAsMywiXFxwaGkiLDJdLFs0LDEsIiIsMCx7InN0eWxlIjp7ImhlYWQiOnsibmFtZSI6Im5vbmUifX19XSxbNSwwLCIiLDAseyJzdHlsZSI6eyJoZWFkIjp7Im5hbWUiOiJub25lIn19fV0sWzYsMCwiIiwwLHsic3R5bGUiOnsiaGVhZCI6eyJuYW1lIjoibm9uZSJ9fX1dXQ==
\begin{tikzcd}[row sep={between origins,20mm},column sep={between origins,5mm}]
	{N_Y Y_2} &[3mm]&[3mm] {\frac{N_Y X}{N_Y Y_1}} &[10mm]&[3mm] {N_{Y_1}X} \\[-9mm]
	& Y && Y_1 \\
	& Y_2 && X
	\arrow["f", from=3-2, to=3-4]
	\arrow["\varphi", from=2-4, to=3-4]
	\arrow["g", from=2-2, to=2-4]
	\arrow["\phi"', from=2-2, to=3-2]
	\arrow[no head, from=1-5, to=2-4]
	\arrow[no head, from=1-3, to=2-2]
	\arrow[no head, from=1-1, to=2-2]
\end{tikzcd}
\end{equation}
Assume that $TY_1\big|_{Y} \cap TY_2\big|_{Y} = TY$, we get excess intersection formula:
$$[Y_1]_{X}^G \otimes [Y_2]_{X}^G = \iota_{Y,*} \left(\eu( \mathcal{T} ) \cdot [Y]_{Y}^G \right).$$
In particular, when $Y=\pt$ is a point, we get simplified formula in $K^G(X)$:
$$[Y_1]^G \otimes [Y_2]^G = \eu( \mathcal{T} ) \cdot [Y]^G $$
where $\eu( \mathcal{T} ) \in \Rpt(G)$ acts by scalar multiplication.
\end{theorem}

Readers may find Theorem \ref{thm:excess_intersection_formula} as a special case of excess base change theorem. In fact,
\begin{equation*}
\begin{aligned}\
   [Y_1]_{X}^G \otimes [Y_2]_{X}^G=\;& [Y_1]_{X}^G \otimes f_*[Y_2]_{Y_2}^G&& \text{definition of $[Y_2]_{X}^G$}\\
   =\;& f_*\left(f^*[Y_1]_{X}^G \otimes [Y_2]_{Y_2}^G\right) && \text{proper projection formula}\\
   =\;& f_*\left(f^*[Y_1]_{X}^G \right) && \text{Lemma \ref{lem:unit_of_tensor_product}}\\
   =\;& f_*\left(f^*\varphi_{*}[Y_1]_{Y_1}^G \right) && \text{definition of $[Y_1]_{X}^G$}\\
   =\;& f_*\left(\phi_{*} \left( \eu(\mathcal{T})\cdot g^*[Y_1]_{Y_1}^G\rule{0mm}{3.6mm}  \right)\rule{0mm}{4mm} \right) && \text{excess base change to \eqref{eq:excess_intersection_formula}}\\
   =\;& \iota_{Y,*} \left(\eu( \mathcal{T} ) \cdot [Y]_{Y}^G \right) &&\\
\end{aligned}
\end{equation*}

The projection formula is stated here.
\begin{proposition}[Projection formula]
For any proper $G$-equivariant morphism $f:Y \longrightarrow X$ of globally finite $\Tor$-dimension, $\alpha \in K^G(Y)$, $\beta \in K^G(X)$, we have proper projection formula:
$$f_* \alpha \otimes \beta =f_*(\alpha \otimes f^{*} \beta).$$
\end{proposition}

\section{Application: convolution structure}
In this section, we will apply Definition \ref{def:convolution_product} and Theorem \ref{thm:excess_intersection_formula} to our typical varieties. In particular, we will get the convolution product formula in terms of basis elements $\preimage{\phi}_{\ww}$ and $\preimage{\phi}_{\ww,\ww'}$.

\subsection{Algebraic structures induced by convolution product}

\begin{defn}[Convolution product sturcture on $K^{G_{\dimvec{d}}}(\St_{\dimvec{d}})$]
Following notations in \ref{def:convolution_product}, We take $G=G_{\dimvec{d}}$,
\begin{equation*}
\begin{aligned}
  M_1=\;& M_2=M_3= \RRep_{\dimvec{d}}(Q) \\ 
  Z_{12}=\;& Z_{23}= \St_{\dimvec{d}} \\ 
  \St_{\dimvec{d}}=\;& \RRep_{\dimvec{d}}(Q) \times_{\Rep_{\dimvec{d}}(Q)} \RRep_{\dimvec{d}}(Q) \subseteq \RRep_{\dimvec{d}}(Q) \times \RRep_{\dimvec{d}}(Q)
\end{aligned}
\end{equation*}
By definition, we see that $\St_{\dimvec{d}} \circ \St_{\dimvec{d}} = \St_{\dimvec{d}}$. Therefore, we define a ring structure on $K^{G_{\dimvec{d}}}(\St_{\dimvec{d}})$:
$$\fakestar: K^{G_{\dimvec{d}}}(\St_{\dimvec{d}}) \times K^{G_{\dimvec{d}}}(\St_{\dimvec{d}}) \longrightarrow K^{G_{\dimvec{d}}}(\St_{\dimvec{d}}).$$
\end{defn}

\begin{defn}[$K^{G_{\dimvec{d}}}(\St_{\dimvec{d}})$-module sturcture on $K^{G_{\dimvec{d}}}(\RRep_{\dimvec{d}}(Q))$]
Following notations in \ref{def:convolution_product}, We take $G=G_{\dimvec{d}}$,
\begin{equation*}
\begin{aligned}
  M_1=\;& M_2= \RRep_{\dimvec{d}}(Q)  \qquad& M_3=\;& \{\pt\}\\ 
  Z_{12}=\;& \St_{\dimvec{d}} & Z_{23}=\;& \RRep_{\dimvec{d}}(Q) \\ 
\end{aligned}
\end{equation*}
By definition, we see that $\St_{\dimvec{d}} \circ \RRep_{\dimvec{d}}(Q) = \RRep_{\dimvec{d}}(Q)$. Therefore, we define a $K^{G_{\dimvec{d}}}(\St_{\dimvec{d}})$-module sturcture on $K^{G_{\dimvec{d}}}\left(\RRep_{\dimvec{d}}(Q)\right)$:
$$\star: K^{G_{\dimvec{d}}}(\St_{\dimvec{d}}) \times K^{G_{\dimvec{d}}}\left(\RRep_{\dimvec{d}}(Q)\right) \longrightarrow K^{G_{\dimvec{d}}}\left(\RRep_{\dimvec{d}}(Q)\right).$$
\end{defn}

\begin{remark}\label{rmk:process_convolution_product}
Notice that in the construction of the convolution product, pullback, tensor product and proper pushforward are compatible with the forgetful map of groups. Therefore, the following diagrams commute:
% https://q.uiver.app/?q=WzAsMTgsWzAsMCwiS157R197XFxkaW12ZWN7ZH19fShcXFN0X3tcXGRpbXZlY3tkfX0pIl0sWzEsMCwiS157R197XFxkaW12ZWN7ZH19fShcXFN0X3tcXGRpbXZlY3tkfX0pIl0sWzIsMCwiS157R197XFxkaW12ZWN7ZH19fShcXFN0X3tcXGRpbXZlY3tkfX0pIl0sWzMsMCwiS157R197XFxkaW12ZWN7ZH19fShcXFN0X3tcXGRpbXZlY3tkfX0pIl0sWzQsMCwiS157R197XFxkaW12ZWN7ZH19fVxcbGVmdChcXFJSZXBfe1xcZGltdmVje2R9fShRKVxccmlnaHQpIl0sWzUsMCwiS157R197XFxkaW12ZWN7ZH19fVxcbGVmdChcXFJSZXBfe1xcZGltdmVje2R9fShRKVxccmlnaHQpIl0sWzAsMSwiS157VF97XFxkaW12ZWN7ZH19fShcXFN0X3tcXGRpbXZlY3tkfX0pIl0sWzEsMSwiS157VF97XFxkaW12ZWN7ZH19fShcXFN0X3tcXGRpbXZlY3tkfX0pIl0sWzIsMSwiS157VF97XFxkaW12ZWN7ZH19fShcXFN0X3tcXGRpbXZlY3tkfX0pIl0sWzMsMSwiS157VF97XFxkaW12ZWN7ZH19fShcXFN0X3tcXGRpbXZlY3tkfX0pIl0sWzQsMSwiS157VF97XFxkaW12ZWN7ZH19fVxcbGVmdChcXFJSZXBfe1xcZGltdmVje2R9fShRKVxccmlnaHQpIl0sWzUsMSwiS157VF97XFxkaW12ZWN7ZH19fVxcbGVmdChcXFJSZXBfe1xcZGltdmVje2R9fShRKVxccmlnaHQpIl0sWzAsMiwiXFxLY3VybF57VF97XFxkaW12ZWN7ZH19fShcXFN0X3tcXGRpbXZlY3tkfX0pIl0sWzEsMiwiXFxLY3VybF57VF97XFxkaW12ZWN7ZH19fShcXFN0X3tcXGRpbXZlY3tkfX0pIl0sWzIsMiwiXFxLY3VybF57VF97XFxkaW12ZWN7ZH19fShcXFN0X3tcXGRpbXZlY3tkfX0pIl0sWzMsMiwiXFxLY3VybF57VF97XFxkaW12ZWN7ZH19fShcXFN0X3tcXGRpbXZlY3tkfX0pIl0sWzQsMiwiXFxLY3VybF57VF97XFxkaW12ZWN7ZH19fVxcbGVmdChcXFJSZXBfe1xcZGltdmVje2R9fShRKVxccmlnaHQpIl0sWzUsMiwiXFxLY3VybF57VF97XFxkaW12ZWN7ZH19fVxcbGVmdChcXFJSZXBfe1xcZGltdmVje2R9fShRKVxccmlnaHQpIl0sWzAsNiwiIiwwLHsic3R5bGUiOnsidGFpbCI6eyJuYW1lIjoiaG9vayIsInNpZGUiOiJib3R0b20ifX19XSxbMSw3LCIiLDAseyJzdHlsZSI6eyJ0YWlsIjp7Im5hbWUiOiJob29rIiwic2lkZSI6ImJvdHRvbSJ9fX1dLFsyLDgsIiIsMCx7InN0eWxlIjp7InRhaWwiOnsibmFtZSI6Imhvb2siLCJzaWRlIjoiYm90dG9tIn19fV0sWzMsOSwiIiwwLHsic3R5bGUiOnsidGFpbCI6eyJuYW1lIjoiaG9vayIsInNpZGUiOiJib3R0b20ifX19XSxbNCwxMCwiIiwwLHsic3R5bGUiOnsidGFpbCI6eyJuYW1lIjoiaG9vayIsInNpZGUiOiJib3R0b20ifX19XSxbNiwxMiwiIiwwLHsic3R5bGUiOnsidGFpbCI6eyJuYW1lIjoiaG9vayIsInNpZGUiOiJib3R0b20ifX19XSxbNywxMywiIiwwLHsic3R5bGUiOnsidGFpbCI6eyJuYW1lIjoiaG9vayIsInNpZGUiOiJib3R0b20ifX19XSxbOCwxNCwiIiwwLHsic3R5bGUiOnsidGFpbCI6eyJuYW1lIjoiaG9vayIsInNpZGUiOiJib3R0b20ifX19XSxbOSwxNSwiIiwwLHsic3R5bGUiOnsidGFpbCI6eyJuYW1lIjoiaG9vayIsInNpZGUiOiJib3R0b20ifX19XSxbMTAsMTYsIiIsMCx7InN0eWxlIjp7InRhaWwiOnsibmFtZSI6Imhvb2siLCJzaWRlIjoiYm90dG9tIn19fV0sWzUsMTEsIiIsMCx7InN0eWxlIjp7InRhaWwiOnsibmFtZSI6Imhvb2siLCJzaWRlIjoiYm90dG9tIn19fV0sWzExLDE3LCIiLDAseyJzdHlsZSI6eyJ0YWlsIjp7Im5hbWUiOiJob29rIiwic2lkZSI6ImJvdHRvbSJ9fX1dLFs0LDUsIlxcZGl2aWRlb250aW1lcyJdLFsxMCwxMSwiXFxkaXZpZGVvbnRpbWVzIl0sWzE2LDE3LCJcXGRpdmlkZW9udGltZXMiXSxbMSwyLCIqIl0sWzcsOCwiKiJdLFsxMywxNCwiKiJdLFszLDQsIlxcdGltZXMiLDEseyJzdHlsZSI6eyJib2R5Ijp7Im5hbWUiOiJub25lIn0sImhlYWQiOnsibmFtZSI6Im5vbmUifX19XSxbOSwxMCwiXFx0aW1lcyIsMSx7InN0eWxlIjp7ImJvZHkiOnsibmFtZSI6Im5vbmUifSwiaGVhZCI6eyJuYW1lIjoibm9uZSJ9fX1dLFsxNSwxNiwiXFx0aW1lcyIsMSx7InN0eWxlIjp7ImJvZHkiOnsibmFtZSI6Im5vbmUifSwiaGVhZCI6eyJuYW1lIjoibm9uZSJ9fX1dLFswLDEsIlxcdGltZXMiLDEseyJzdHlsZSI6eyJib2R5Ijp7Im5hbWUiOiJub25lIn0sImhlYWQiOnsibmFtZSI6Im5vbmUifX19XSxbNiw3LCJcXHRpbWVzIiwxLHsic3R5bGUiOnsiYm9keSI6eyJuYW1lIjoibm9uZSJ9LCJoZWFkIjp7Im5hbWUiOiJub25lIn19fV0sWzEyLDEzLCJcXHRpbWVzIiwxLHsic3R5bGUiOnsiYm9keSI6eyJuYW1lIjoibm9uZSJ9LCJoZWFkIjp7Im5hbWUiOiJub25lIn19fV1d
\[\begin{tikzcd}
	{K^{G_{\dimvec{d}}}(\St_{\dimvec{d}})} &[-9mm] {K^{G_{\dimvec{d}}}(\St_{\dimvec{d}})} &[-3mm] {K^{G_{\dimvec{d}}}(\St_{\dimvec{d}})} &[-3mm] {K^{G_{\dimvec{d}}}(\St_{\dimvec{d}})} &[-9mm] {K^{G_{\dimvec{d}}}\left(\RRep_{\dimvec{d}}(Q)\right)} &[-3mm] {K^{G_{\dimvec{d}}}\left(\RRep_{\dimvec{d}}(Q)\right)} \\
	{K^{T_{\dimvec{d}}}(\St_{\dimvec{d}})} & {K^{T_{\dimvec{d}}}(\St_{\dimvec{d}})} & {K^{T_{\dimvec{d}}}(\St_{\dimvec{d}})} & {K^{T_{\dimvec{d}}}(\St_{\dimvec{d}})} & {K^{T_{\dimvec{d}}}\left(\RRep_{\dimvec{d}}(Q)\right)} & {K^{T_{\dimvec{d}}}\left(\RRep_{\dimvec{d}}(Q)\right)} \\
	{\Kcurl^{T_{\dimvec{d}}}(\St_{\dimvec{d}})} & {\Kcurl^{T_{\dimvec{d}}}(\St_{\dimvec{d}})} & {\Kcurl^{T_{\dimvec{d}}}(\St_{\dimvec{d}})} & {\Kcurl^{T_{\dimvec{d}}}(\St_{\dimvec{d}})} & {\Kcurl^{T_{\dimvec{d}}}\left(\RRep_{\dimvec{d}}(Q)\right)} & {\Kcurl^{T_{\dimvec{d}}}\left(\RRep_{\dimvec{d}}(Q)\right)}
	\arrow[hook', from=1-1, to=2-1]
	\arrow[hook', from=1-2, to=2-2]
	\arrow[hook', from=1-3, to=2-3]
	\arrow[hook', from=1-4, to=2-4]
	\arrow[hook', from=1-5, to=2-5]
	\arrow[hook', from=2-1, to=3-1]
	\arrow[hook', from=2-2, to=3-2]
	\arrow[hook', from=2-3, to=3-3]
	\arrow[hook', from=2-4, to=3-4]
	\arrow[hook', from=2-5, to=3-5]
	\arrow[hook', from=1-6, to=2-6]
	\arrow[hook', from=2-6, to=3-6]
	\arrow["\star", from=1-5, to=1-6]
	\arrow["\star", from=2-5, to=2-6]
	\arrow["\star", from=3-5, to=3-6]
	\arrow["\fakestar", from=1-2, to=1-3]
	\arrow["\fakestar", from=2-2, to=2-3]
	\arrow["\fakestar", from=3-2, to=3-3]
	\arrow["\times"{description}, draw=none, from=1-4, to=1-5]
	\arrow["\times"{description}, draw=none, from=2-4, to=2-5]
	\arrow["\times"{description}, draw=none, from=3-4, to=3-5]
	\arrow["\times"{description}, draw=none, from=1-1, to=1-2]
	\arrow["\times"{description}, draw=none, from=2-1, to=2-2]
	\arrow["\times"{description}, draw=none, from=3-1, to=3-2]
\end{tikzcd}\]
\end{remark}

\begin{defn}[$K^{G_{\dimvec{d}}}(\RRep_{\dimvec{d}}(Q))$-module sturcture on $K^{G_{\dimvec{d}}}(\St_{\dimvec{d}})$]
We know that 
$$\RRep_{\dimvec{d}}(Q) \cong \St_{\Id} \subseteq \St_{\dimvec{d}}, \qquad \St_{\Id} \circ \St_{\Id} = \St_{\Id},$$
so $K^{G_{\dimvec{d}}}\left(\RRep_{\dimvec{d}}(Q)\right)$ can be realized as a $R(G_{\dimvec{d}})$-subalgebra of $K^{G_{\dimvec{d}}}(\St_{\dimvec{d}})$, and $K^{G_{\dimvec{d}}}(\St_{\dimvec{d}})$ has the $K^{G_{\dimvec{d}}}\left(\RRep_{\dimvec{d}}(Q)\right)$-module structure induced by the convolution product:
$$\fakestar: K^{G_{\dimvec{d}}}\left(\RRep_{\dimvec{d}}(Q)\right) \times K^{G_{\dimvec{d}}}(\St_{\dimvec{d}}) \longrightarrow K^{G_{\dimvec{d}}}(\St_{\dimvec{d}}).$$
\end{defn}

\subsection{Convolution product formula}\label{subsec:convolution_product_fml}


In this subsection, we compute the convolution product in the bottom line of the diagram in Remark \ref{rmk:process_convolution_product}.
\begin{proposition}[Convolution product formula]\label{prop:convolution_product_formula}
For $\ww$, $\ww'$, $\ww''$, $\ww''' \in \WWd$, we have
\begin{equation*}
\begin{aligned}
  \preimage{\psi}_{\ww,\ww'} \fakestar \preimage{\psi}_{\ww'',\ww'''}=\;& \delta_{\ww',\ww''}\preimage{\Lambda}_{\ww'} \preimage{\psi}_{\ww,\ww'''} \\ 
  \preimage{\psi}_{\ww,\ww'} \star \preimage{\psi}_{\ww''\phantom{,\ww'''}}=\;& \delta_{\ww',\ww''}\preimage{\Lambda}_{\ww'} \preimage{\psi}_{\ww}. \\   
\end{aligned}
\end{equation*}
\end{proposition}
\begin{proof}
Follow the Definition \ref{def:convolution_product} and Theorem \ref{thm:excess_intersection_formula} if needed.

For clearance, we divide the proof into $4$ cases.

\paragraph*{\underline{Case 1.}}Assume $\ww' \neq \ww''$, need to show $\preimage{\psi}_{\ww,\ww'} \fakestar \preimage{\psi}_{\ww'',\ww'''}=0$.\\[-3mm]

Denote \footnote{For some people, the notation
$$Y_{12}:= \left\{\rule{0mm}{3mm}\!\left(\rule{0mm}{2.8mm}(\rho_0, F_{\ww}),(\rho_0, F_{\ww'})\right) \right\} \subseteq \St_{\dimvec{d}}$$ is better for understanding. We don't write like that, because too many brackets occupy attentions.
}
$$Y_{12}:= \{(\rho_0, F_{\ww}, F_{\ww'}) \} \subseteq \St_{\dimvec{d}}, \qquad Y_{23}:= \{(\rho_0, F_{\ww''}, F_{\ww'''}) \} \subseteq \St_{\dimvec{d}}.$$
Since $\ww' \neq \ww''$, $p_{12}^{-1} (Y_{12}) \cap
p_{23}^{-1} (Y_{23}) = \varnothing$, so
\begin{equation*}
\begin{aligned}
\preimage{\psi}_{\ww,\ww'} \fakestar \preimage{\psi}_{\ww'',\ww'''}=\;& \left[ Y_{12}   \right]_{\St_{\dimvec{d}}}^{T_{\dimvec{d}}}  \fakestar \left[ Y_{23}  \right]_{\St_{\dimvec{d}}}^{T_{\dimvec{d}}}\\ 
=\;&p_{13,*}\left(p_{12}^* \left[ Y_{12}   \right]_{M_{12}}^{T_{\dimvec{d}}}  \otimes p_{23}^* \left[ Y_{23}  \right]_{M_{23}}^{T_{\dimvec{d}}} \right)\\
=\;&p_{13,*}\left(\left[ p_{12}^{-1}(Y_{12})   \right]_{M_{123}}^{T_{\dimvec{d}}}  \otimes \left[  p_{12}^{-1}(Y_{23})  \right]_{M_{123}}^{T_{\dimvec{d}}} \right)\\
=\;&0
\end{aligned}
\end{equation*}

\paragraph*{\underline{Case 2.}}Assume $\ww' \neq \ww''$, need to show $\preimage{\psi}_{\ww,\ww'} \star \preimage{\psi}_{\ww''}=0$.\\[-3mm]

Denote
$$Y_{12}:= \{(\rho_0, F_{\ww}, F_{\ww'}) \} \subseteq \St_{\dimvec{d}}, \qquad Y_{23}:= \{(\rho_0, F_{\ww''}) \} \subseteq \RRep_{\dimvec{d}}(Q).$$
Since $\ww' \neq \ww''$, $p_{12}^{-1} (Y_{12}) \cap
p_{23}^{-1} (Y_{23}) = \varnothing$, so
\begin{equation*}
\begin{aligned}
\preimage{\psi}_{\ww,\ww'} \star \preimage{\psi}_{\ww''}=\;& \left[ Y_{12}   \right]_{\St_{\dimvec{d}}}^{T_{\dimvec{d}}}  \star \left[ Y_{23}  \right]_{\RRep_{\dimvec{d}}(Q)}^{T_{\dimvec{d}}}\\ 
=\;&p_{13,*}\left(p_{12}^* \left[ Y_{12}   \right]_{M_{12}}^{T_{\dimvec{d}}}  \otimes p_{23}^* \left[ Y_{23}  \right]_{M_{23}}^{T_{\dimvec{d}}} \right)\\
=\;&p_{13,*}\left(\left[ p_{12}^{-1}(Y_{12})   \right]_{M_{123}}^{T_{\dimvec{d}}}  \otimes \left[  p_{12}^{-1}(Y_{23})  \right]_{M_{123}}^{T_{\dimvec{d}}} \right)\\
=\;&0
\end{aligned}
\end{equation*}

\paragraph*{\underline{Case 3.}}For $\ww$, $\ww'$, $\ww'' \in \WWd$, need to show that $$\preimage{\psi}_{\ww,\ww'} \fakestar \preimage{\psi}_{\ww',\ww''}=\preimage{\Lambda}_{\ww'}\preimage{\psi}_{\ww,\ww''}.$$

Denote 
$$Y_{12}:= \{(\rho_0, F_{\ww}, F_{\ww'}) \} \subseteq \St_{\dimvec{d}}, \qquad Y_{23}:= \{(\rho_0, F_{\ww'}, F_{\ww''}) \} \subseteq \St_{\dimvec{d}},$$
then
\begin{equation*}
\begin{aligned}
  &p_{12}^{-1}(Y_{12})= Y_{12} \times \RRep_{\dimvec{d}}(Q) \qquad & p_{23}^{-1}(Y_{23})=\;&   \RRep_{\dimvec{d}}(Q) \times Y_{23}\\ 
  &p_{12}^{-1}(Y_{12}) \cup p_{23}^{-1}(Y_{23})=Y & Y_{12}\circ Y_{23}=\;&Y_{13},
\end{aligned}
\end{equation*}
where
\begin{equation*}
\begin{aligned}
  &Y=\{y\} \qquad && y=\big( (\rho_0,F_{\ww}),(\rho_0,F_{\ww'}),(\rho_0,F_{\ww''})  \big) & &\in M_{123}  \\ 
  &Y_{13}=\{y_{13}\} \qquad && y_{13}=\big( (\rho_0,F_{\ww}),(\rho_0,F_{\ww''})  \big) & &\in M_{13}  \\
\end{aligned}
\end{equation*}
Therefore, 
\begin{equation*}
\begin{aligned}
\preimage{\psi}_{\ww,\ww'} \fakestar \preimage{\psi}_{\ww',\ww''}=\;& \left[ Y_{12}   \right]_{\St_{\dimvec{d}}}^{T_{\dimvec{d}}}  \fakestar \left[ Y_{23}  \right]_{\St_{\dimvec{d}}}^{T_{\dimvec{d}}}\\ 
=\;&p_{13,*}\left(p_{12}^* \left[ Y_{12}   \right]_{M_{12}}^{T_{\dimvec{d}}}  \otimes p_{23}^* \left[ Y_{23}  \right]_{M_{23}}^{T_{\dimvec{d}}} \right)\\
=\;&p_{13,*}\left(\left[ p_{12}^{-1}(Y_{12})   \right]_{M_{123}}^{T_{\dimvec{d}}}  \otimes \left[  p_{12}^{-1}(Y_{23})  \right]_{M_{123}}^{T_{\dimvec{d}}} \right)\\
=\;&p_{13,*}\left(\eu( \mathcal{T} ) \cdot [Y]_{M_{123}}^{T_{\dimvec{d}}} \right)\\
=\;&\eu( \mathcal{T} )\cdot [Y]_{M_{13}}^{T_{\dimvec{d}}} \\
=\;& \preimage{\Lambda}_{\ww'}\preimage{\psi}_{\ww,\ww''}
\end{aligned}
\end{equation*}
where
$$\mathcal{T}:= \frac{T_y M_{123} }{T_y \big( p_{12}^{-1}(Y_{12}) \big) \oplus T_y \big( p_{23}^{-1}(Y_{23}) \big)} = \frac{\preimage{\mathcal{T}}_{\ww} \oplus \preimage{\mathcal{T}}_{\ww'} \oplus \preimage{\mathcal{T}}_{\ww''}}{\preimage{\mathcal{T}}_{\ww} \hfill\oplus \hfill  \preimage{\mathcal{T}}_{\ww''}}= \preimage{\mathcal{T}}_{\ww'}.$$
\paragraph*{\underline{Case 4.}}For $\ww$, $\ww' \in \WWd$, need to show that $$\preimage{\psi}_{\ww,\ww'} \star \preimage{\psi}_{\ww'}=\preimage{\Lambda}_{\ww'}\preimage{\psi}_{\ww}.$$

Denote 
$$Y_{12}:= \{(\rho_0, F_{\ww}, F_{\ww'}) \} \subseteq \St_{\dimvec{d}}, \qquad Y_{23}:= \{(\rho_0, F_{\ww'}) \} \subseteq \RRep_{\dimvec{d}}(Q),$$
then
\begin{equation*}
\begin{aligned}
  &p_{12}^{-1}(Y_{12})= Y_{12} \times \{\pt\} \qquad & p_{23}^{-1}(Y_{23})=\;&   \RRep_{\dimvec{d}}(Q) \times Y_{23}\\ 
  &p_{12}^{-1}(Y_{12}) \cup p_{23}^{-1}(Y_{23})=Y \qquad& Y_{12}\circ Y_{23}=\;&Y_{13},
\end{aligned}
\end{equation*}
where
\begin{equation*}
\begin{aligned}
  &Y=\{y\} \qquad && y=\big( (\rho_0,F_{\ww}),(\rho_0,F_{\ww'})  \big) & &\in M_{123}  \\ 
  &Y_{13}=\{y_{13}\} \qquad && y_{13}= (\rho_0,F_{\ww}) & &\in M_{13}  \\
\end{aligned}
\end{equation*}
Therefore, 
\begin{equation*}
\begin{aligned}
\preimage{\psi}_{\ww,\ww'} \star \preimage{\psi}_{\ww'}=\;& \left[ Y_{12}   \right]_{\St_{\dimvec{d}}}^{T_{\dimvec{d}}}  \star \left[ Y_{23}  \right]_{\RRep_{\dimvec{d}}(Q)}^{T_{\dimvec{d}}}\\ 
=\;&p_{13,*}\left(p_{12}^* \left[ Y_{12}   \right]_{M_{12}}^{T_{\dimvec{d}}}  \otimes p_{23}^* \left[ Y_{23}  \right]_{M_{23}}^{T_{\dimvec{d}}} \right)\\
=\;&p_{13,*}\left(\left[ p_{12}^{-1}(Y_{12})   \right]_{M_{123}}^{T_{\dimvec{d}}}  \otimes \left[  p_{12}^{-1}(Y_{23})  \right]_{M_{123}}^{T_{\dimvec{d}}} \right)\\
=\;&p_{13,*}\left(\eu( \mathcal{T} ) \cdot [Y]_{M_{123}}^{T_{\dimvec{d}}} \right)\\
=\;&\eu( \mathcal{T} )\cdot [Y]_{M_{13}}^{T_{\dimvec{d}}} \\
=\;& \preimage{\Lambda}_{\ww'}\preimage{\psi}_{\ww}
\end{aligned}
\end{equation*}
where
$$\mathcal{T}:= \frac{T_y M_{123} }{T_y \big( p_{12}^{-1}(Y_{12}) \big) \oplus T_y \big( p_{23}^{-1}(Y_{23}) \big)} = \frac{\preimage{\mathcal{T}}_{\ww} \oplus \preimage{\mathcal{T}}_{\ww'} \oplus 0}{\preimage{\mathcal{T}}_{\ww} \hfill\oplus \hfill  0}= \preimage{\mathcal{T}}_{\ww'}.$$

\end{proof}

Readers can think matrix multiplication as an analog of Proposition \ref{prop:convolution_product_formula}: denote $E_{ij} \in M^{n \times n}(\mathbb{C})$ as the matrix having $1$ in the $(i,j)$-position and $0$ elsewhere, and $e_i \in M^{n \times 1}(\mathbb{C})$ as the standard column vector, then
$$E_{ij} E_{kl}=\delta_{jk}E_{il}\qquad E_{ij}e_k=\delta_{jk}e_i.$$

\subsection{Demazure operator}
In this subsection, we will compute the action of some elements in $K^{G_{\dimvec{d}}}  (\St_{\ftdimvec{d},\ftdimvec{d}'})$ acting on $K^{G_{\dimvec{d}}} \left( \RRep_{\ftdimvec{d}'}(Q) \right)$. As a reminder,
% https://q.uiver.app/?q=WzAsNCxbMCwwLCJLXntHX3tcXGRpbXZlY3tkfX19IFxcbGVmdCggXFxSUmVwX3tcXGZ0ZGltdmVje2R9J30oUSkgXFxyaWdodCkiXSxbMCwxLCJLXntUX3tcXGRpbXZlY3tkfX19IFxcbGVmdCggXFxSUmVwX3tcXGZ0ZGltdmVje2R9J30oUSkgXFxyaWdodCkiXSxbMSwxLCJcXGRpc3BsYXlzdHlsZVxcYmlnb3BsdXNfe3d9IFxcUnB0KFRfe1xcZGltdmVje2R9fSkgXFxsZWZ0WyBcXG92ZXJsaW5le1xccHJlaW1hZ2V7XFxPbWNlbGx9fV97d31ee3V9IFxccmlnaHRdXntUX3tcXGRpbXZlY3tkfX19Il0sWzEsMCwiXFxkaXNwbGF5c3R5bGVcXHBoYW50b217XFxiaWdvcGx1c197d319XFxScHQoVF97XFxkaW12ZWN7ZH19KSBcXGxlZnRbIFxcUlJlcF97XFxmdGRpbXZlY3tkfSd9KFEpIFxccmlnaHRdXntHX3tcXGRpbXZlY3tkfX19Il0sWzMsMl0sWzAsMV0sWzAsMywiXFxjb25nIiwxLHsic3R5bGUiOnsiYm9keSI6eyJuYW1lIjoibm9uZSJ9LCJoZWFkIjp7Im5hbWUiOiJub25lIn19fV0sWzEsMiwiXFxjb25nIiwxLHsic3R5bGUiOnsiYm9keSI6eyJuYW1lIjoibm9uZSJ9LCJoZWFkIjp7Im5hbWUiOiJub25lIn19fV1d
\begin{equation}\label{eq:shift_G_T}
\begin{tikzcd}[column sep={between origins, 38mm}, row sep={between origins, 18mm}]
	{K^{G_{\dimvec{d}}} \left( \RRep_{\ftdimvec{d}'}(Q) \right)} & {\displaystyle\phantom{\bigoplus_{w}}\Rpt(T_{\dimvec{d}}) \left[ \RRep_{\ftdimvec{d}'}(Q) \right]^{G_{\dimvec{d}}}\hspace{-10mm}} \\
	{K^{T_{\dimvec{d}}} \left( \RRep_{\ftdimvec{d}'}(Q) \right)} & {\displaystyle\bigoplus_{w} \Rpt(T_{\dimvec{d}}) \left[ \overline{\preimage{\Omcell}}_{w}^{u} \right]^{T_{\dimvec{d}}}}
	\arrow[from=1-2, to=2-2]
	\arrow[from=1-1, to=2-1]
	\arrow["\cong"{description}, draw=none, from=1-1, to=1-2]
	\arrow["\cong"{description}, draw=none, from=2-1, to=2-2]
\end{tikzcd}
\end{equation}
where the $R(T_{\dimvec{d}})$-module structure on $K^{G_{\dimvec{d}}} \left( \RRep_{\ftdimvec{d}'}(Q) \right)$ is induced by the induction formula.

For $f \in \Rpt(T_{\dimvec{d}}) \cong \mathbb{Z}\!\left[ e_1^{\pm 1},\ldots,e_{\abdimvec{d}}^{\pm 1} \right]$, denote $f^{u}:=f \cdot \left[ \RRep_{\ftdimvec{d}'}(Q) \right]^{G_{\dimvec{d}}}$. Under the morphism \eqref{eq:shift_G_T}, $f$ is sent to $f \cdot \left[ \RRep_{\ftdimvec{d}'}(Q) \right]^{T_{\dimvec{d}}}$. Viewing $f^{u}$ as an element in $\Kcurl^{G_{\dimvec{d}}} \left( \RRep_{\ftdimvec{d}'}(Q) \right)$, we get
$$f^{u}=\sum_{w} f(e_1,\ldots,e_{\abdimvec{d}}) \preimage{\Lambda}_{wu}^{-1} \preimage{\psi}_{wu}.$$
\begin{remark}\label{rmk:relabling_of_coef}
This formula looks not so compatible with the group action. To facilitate our computation, we relable the coefficient ring before $\preimage{\psi}_{\ww}$ by $e_i^{\ww}:= e_{\ww^{-1}(i)}$, which means that
$$\Kcurl^{T_{\dimvec{d}}} \left(\RRep_{\dimvec{d}}(Q)\right) \cong \bigoplus_{\ww} \mathbb{Z}\!\left[ e_1^{\ww,\pm 1},\ldots,e_{\abdimvec{d}}^{\ww,\pm 1} \right] \preimage{\psi}_{\ww}$$
Therefore, 
\begin{equation*}
\begin{aligned}
 f^{u} =\;&  \sum_{w} (wuf)(e_1^{wu},\ldots,e_{\abdimvec{d}}^{wu}) \preimage{\Lambda}_{wu}^{-1} \preimage{\psi}_{wu}.\\ 
 \hat{=}\;&  \sum_{w} (wuf) \; \preimage{\Lambda}_{wu}^{-1} \preimage{\psi}_{wu}.\\ 
\end{aligned}
\end{equation*}
Later, every expression before $\preimage{\psi}_{\ww}$ should be viewed as an expression in $\mathbb{Z}\!\left[ e_1^{\ww,\pm 1},\ldots,e_{\abdimvec{d}}^{\ww,\pm 1} \right]$.
\end{remark}

\begin{defn}[Demazure operator]
For $i \in \{ 1,\ldots, \abdimvec{d}-1 \}$, the (absolute) Demazure operator is defined as
$$D_i:= \left[\St_{s_i} \right]^{G_{\dimvec{d}}} \in K^{G_{\dimvec{d}}}(\St_{\dimvec{d}}).$$
View $D_i$ as an element in $\Kcurl^{T_{\dimvec{d}}}(\St_{\dimvec{d}})$, we get
$$D_i=\sum_{\ww \in \WWd} \left( \preimage{\Lambda}_{\ww,\ww s}^{s} \right)^{-1} \preimage{\psi}_{\ww,\ww s} + \sum_{\stackrel{\ww \in \WWd}{\ww s \ww^{-1}\in \Wd}} \left( \preimage{\Lambda}_{\ww,\ww}^{s} \right)^{-1} \preimage{\psi}_{\ww,\ww}.$$

We also have the relative version. Suppose that $\Wd us_i=\Wd u'$ (which guarantees the existence of $\St_{s_i}^{u,u'}$), the (relative) Demazure operator is defined as
$$D_i^{u,u'}:= \left[\St_{s_i}^{u,u'} \right]^{G_{\dimvec{d}}} \in K^{G_{\dimvec{d}}}(\St^{u,u'}).$$
View $D_i^{u,u'}$ as an element in $\Kcurl^{T_{\dimvec{d}}}(\St^{u,u'})$, we get
$$D_i^{u,u'}=\sum_{w} \left( \preimage{\Lambda}_{wu,wu s}^{s} \right)^{-1} \preimage{\psi}_{wu,wu s} + \delta_{u,u'}\sum_{w} \left( \preimage{\Lambda}_{wu,wu}^{s} \right)^{-1} \preimage{\psi}_{wu,wu}.$$

The equivariant cohomology theory version of Demazure operators are denoted by $\partial_i$ and $\partial_i^{u,u'}$.
\end{defn}
\begin{theorem}\label{thm:Demazure_operator_1}
We have a formula of Demazure operator:
\begin{equation*}\label{eq:Demazure_operator}
D_i^{u,u'} \star f^{u'}=\begin{cases}
\left[\left( \raisebox{2mm}{$\displaystyle\frac{sf}{ 1-\frac{e_{u(i+1)}^{u}}{e_{u(i)}^{u}}}     + \frac{f}{1-\frac{e_{u(i)}^{u}}{e_{u(i+1)}^{u}}}$}  \right)\left(\displaystyle 1-\frac{e_{u(i+1)}^{u}}{e_{u(i)}^{u}}\right)^{k} \right]^{u} & u=u',\\[8mm]
\left[sf  \left(\displaystyle 1-\frac{e_{u(i+1)}^{u}}{e_{u(i)}^{u}}\right)^{k-1} \right]^{u} & u \neq u'.
\end{cases}
\end{equation*}
and similar for the equivariant cohomology theory:
\begin{equation*}\label{eq:Demazure_operator_cth}
\partial_i^{u,u'} \star f^{u'}=\begin{cases}
\left[\left( \displaystyle\frac{sf}{ \lambda_{u(i+1)}^{u}-\lambda_{u(i)}^{u}}     + \frac{f}{\lambda_{u(i)}^{u}-\lambda_{u(i+1)}^{u} }  \right)\left(\lambda_{u(i+1)}^{u}-\lambda_{u(i)}^{u}\right)^{k} \right]^{u} & u=u',\\[8mm]
\left[sf  \left(\lambda_{u(i+1)}^{u}-\lambda_{u(i)}^{u}\right)^{k-1} \right]^{u} & u \neq u'.
\end{cases}
\end{equation*}
\end{theorem}

In the computation we mainly focus on the $K$-theory. Using \ref{rmk:relabling_of_coef}, one can compute $D_i^{u,u'} \star f^{u'}$ in terms of $\phi$'s: ($s:=s_i$ for simplicity)
\begingroup
%\allowdisplaybreaks
\begin{align*}
  D_i^{u,u'} \star f^{u'}=\;& \left(\sum_{w} \left( \preimage{\Lambda}_{wu,wu s}^{s} \right)^{-1} \preimage{\psi}_{wu,wu s} + \delta_{u,u'}\sum_{w} \left( \preimage{\Lambda}_{wu,wu}^{s} \right)^{-1} \preimage{\psi}_{wu,wu}\right) \\ 
  &\hspace{70mm}\times \left(\sum_{w} (wu'f) \; \preimage{\Lambda}_{wu'}^{-1} \preimage{\psi}_{wu'}\right)\\
  =\;& \left(\sum_{w} \left( \preimage{\Lambda}_{wu,wu s}^{s} \right)^{-1} \preimage{\psi}_{wu,wu s}\right) \cdot \left(\sum_{w} (wusf) \; \preimage{\Lambda}_{wus}^{-1} \preimage{\psi}_{wus}\right)\\
  & +\delta_{u,u'}\left(\sum_{w} \left( \preimage{\Lambda}_{wu,wu}^{s} \right)^{-1} \preimage{\psi}_{wu,wu}\right) \cdot \left(\sum_{w} (wuf) \; \preimage{\Lambda}_{wu}^{-1} \preimage{\psi}_{wu}\right)\\
  =\;& \left(\sum_{w} (wusf) \left( \preimage{\Lambda}_{wu,wu s}^{s} \right)^{-1} \preimage{\psi}_{wu} \right) +\delta_{u,u'}\left(\sum_{w} (wuf) \left( \preimage{\Lambda}_{wu,wu}^{s} \right)^{-1} \preimage{\psi}_{wu} \right)\\
  =\;& \sum_{w}
  \left[ \left(\frac{wusf}{\preimage{\Lambda}_{wu,wu s}^{s}}     +\delta_{u,u'} \frac{wuf}{\preimage{\Lambda}_{wu,wu}^{s}} \right) \preimage{\Lambda}_{wu} \right]\preimage{\Lambda}_{wu}^{-1}\preimage{\psi}_{wu} \\
  =\;& \sum_{w} w
  \left[  \left(\frac{usf}{\preimage{\Lambda}_{u,u s}^{s}}     +\delta_{u,u'} \frac{uf}{\preimage{\Lambda}_{u,u}^{s}} \right) \preimage{\Lambda}_{u} \right]\preimage{\Lambda}_{wu}^{-1}\preimage{\psi}_{wu} \\
  =\;& \sum_{w} wu
    \left[  \left(\frac{sf}{u^{-1}\preimage{\Lambda}_{u,u s}^{s}}     +\delta_{u,u'} \frac{f}{u^{-1}\preimage{\Lambda}_{u,u}^{s}} \right) u^{-1}\preimage{\Lambda}_{u} \right]\preimage{\Lambda}_{wu}^{-1}\preimage{\psi}_{wu} \\
  =\;& 
        \left[  \left(\frac{sf}{u^{-1}\preimage{\Lambda}_{u,u s}^{s}}     +\delta_{u,u'} \frac{f}{u^{-1}\preimage{\Lambda}_{u,u}^{s}} \right) u^{-1}\preimage{\Lambda}_{u} \right]^{u}    
\end{align*}
\endgroup
Recall Subsection \ref{subsec:tangent_space} (especially Proposition  \ref{prop:tangent_space_2}), we get
$$
\preimage{\mathcal{T}}_{u,us}^{s}\cong \mathfrak{r}_{u,us} \oplus \mathfrak{n}_{u}^{-} \oplus \mathfrak{m}_{u,us}
\qquad
\preimage{\mathcal{T}}_{u,u}^{s}\cong  \mathfrak{r}_{u,us} \oplus \mathfrak{n}_{u}^{-} \oplus \mathfrak{m}_{us,u}
\qquad
\preimage{\mathcal{T}}_{u} \cong \mathfrak{r}_{u} \oplus \mathfrak{n}_{u}^{-}. $$
Therefore, 
\begin{equation}\label{eq:final_destination}
D_i^{u,u'} \star f^{u'}=\left[\left(\frac{sf}{u^{-1}\eu(\mathfrak{m}_{u,us})}     +\delta_{u,u'} \frac{f}{u^{-1}\eu(\mathfrak{m}_{us,u})}  \right)u^{-1}\eu(\mathfrak{d}_{u,us}) \right]^{u}.
\end{equation}
Recall the computation in \ref{eg:5-case-3} and Section \ref{sec:euler_class}. We collect needed information in Table \ref{table:euler_class_1}:

\begin{table}[ht]
  \vspace{0cm}
    \centering  \includegraphics[width=15cm]{figures/table/table_euler_class.pdf}
      \caption{}
      \label{table:euler_class_1}        
\end{table}

In the table, $\lambda_l^{u}:=\lambda_{u^{-1}(l)}$, and $k$ stands for the number of arrows from the vertex associated to $v_{u_{i+1}}$ to the vertex associated to $v_{u_{i}}$.

Theorem \ref{thm:Demazure_operator_1} is our final destination in this part. We will express its importance in Subsection \ref{subsec:miscellaneous}, see some generalizations in Chapter \ref{chap:generalization} and compute some examples in Chapter \ref{chap:diagram}.

\subsection{Miscellaneous}\label{subsec:miscellaneous}
In this subsection, we collect some results which are of significant importance theoretically. The arguments in reference work for both $K$-theory and cohomology theory.

\begin{proposition}
The action of $K^{G_{\dimvec{d}}}(\St_{\dimvec{d}})$ on $K^{G_{\dimvec{d}}}(\RRep_{\dimvec{d}}(Q))$ is faithful.
\end{proposition}
\begin{proof}[Sketch of proof]
Reduce the problem to the faithfulness for the action of $\Kcurl^{T_{\dimvec{d}}}(\St_{\dimvec{d}})$ on $\Kcurl^{T_{\dimvec{d}}}(\RRep_{\dimvec{d}}(Q))$. For details, see \cite[Theorem 10.10]{przezdziecki2015geometric}.
\end{proof}

\begin{proposition}
The elements $\{D_{i}^{u,u'}\}$ generate $K^{G_{\dimvec{d}}}(\St_{\dimvec{d}})$ as a $K^{G_{\dimvec{d}}}(\RRep_{\dimvec{d}}(Q))$-algebra.
\end{proposition}
\begin{proof}[Sketch of proof]
See \cite[Theorem 11.3]{przezdziecki2015geometric}. The key observation is \cite[Lemma 7.30, 11.4]{przezdziecki2015geometric}.
\end{proof}
Combining these propositions with Theorem \ref{thm:Demazure_operator_1}, we understand the convolution structure of $K^{G_{\dimvec{d}}}(\St_{\dimvec{d}})$ theoretically.
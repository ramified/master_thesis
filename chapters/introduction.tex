\chapter*{Introduction}
\addcontentsline{toc}{chapter}{Introduction}

With two different goals, this master thesis is naturally divided into two parts. The first part is dedicated to computing the equivariant $K$-theory of Steinberg varieties, while the second part is dedicated to constructing affine pavings for quiver partial flag varieties.

\section*{Part I}

The Steinberg variety $\St$ was introduced in \cite{steinberg1976desingularization} and, in type $A$, consists of triples of a nilpotent operator and two flags fixed by this operator. The (top) Borel--Moore homology of $\St$ can be equipped with the structure of convolution algebra which yields the group algebra of Weyl group, see \cite{kazhdan1980topological}.
 
Similarly, Lusztig showed in \cite{lusztig1985equivariant} that the affine Hecke algebra can realized as the $G \times \mathbb{C}^{\times}$-equivariant $K$-theory of $\St$—a fact that is often referred to as the Kazhdan–Lusztig isomorphism. Moreover, irreducible representations of these algebras can be constructed using the geometry of Springer fibers.

Khovanov and Lauda \cite{https://doi.org/10.48550/arxiv.0803.4121} and Rouquier \cite{https://doi.org/10.48550/arxiv.0812.5023} defined an algebra, called the quiver Hecke algebra or KLR algebra, in order to categorify quantum groups. In fact, Varagnolo--Vasserot \cite{varagnolo2011canonical} show that this algebra arises as the $G$-equivariant cohomology of a quiver version $\St_{\dimvec{d}}$ of the Steinberg variety.

Motivated by this, we study a $K$-theoretic version of the KLR algebra in the first part of this thesis. We follow methods from Varagnolo--Vasserot \cite{varagnolo2011canonical} and Przezdziecki’s master thesis \cite{przezdziecki2015geometric}. We base our $K$-theoretic arguments on the exposition in \cite[Chapter 5]{chriss1997representation}.

$\,$

To state our results, we fix some notation. For a $G$-variety $X$, denote by $K^G(X)=K\big(\!\Coh^G(X)\big)$ the Grothendieck group of $G$-equivariant coherent sheaves on $X$, and by $\Rpt(G)=K^G(\pt)$ the representation ring. Fix a quiver $Q$ (without loops and cycles) and dimension vector $\dimvec{d}$. Denote by $G_{\dimvec{d}}=\prod_{i \in v(Q)} \GL_{\dimvec{d}_i}(\mathbb{C})$ and by $T_{\dimvec{d}} \subset G_{\dimvec{d}}$ the maximal torus of diagonal matrices. Let $\mathcal{F}_{\dimvec{d}}$ be the complete flag variety and $\Rep_{\dimvec{d}}(Q)$ be the vector space of representations of $Q$ with dimension vector $\dimvec{d}$.

We will study the geometry of the incidence varieties $\RRep_{\dimvec{d}}(Q) \subseteq \Rep_{\dimvec{d}}(Q) \times \mathcal{F}_{\dimvec{d}}$ and $\St_{\dimvec{d}} \subseteq \Rep_{\dimvec{d}}(Q) \times \mathcal{F}_{\dimvec{d}} \times \mathcal{F}_{\dimvec{d}}$ consisting of pairs, respectively triples, of a representation of $Q$ and one, respectively two, flags fixed by this representation. Now we state Theorem \ref{thm:A} as the main result in Part \ref{part:Steinberg_varieties}.

\begin{theoremA}\label{thm:A}
Under the convolution product, $K^{G_{\dimvec{d}}}(\St_{\dimvec{d}})$ has a $K^{G_{\dimvec{d}}}\left(\RRep_{\dimvec{d}}(Q)\right) \cong \Rpt(T_{\dimvec{d}})$-algebra structure. Moreover,
\begingroup
\upshape
\setlist{itemsep=-0.4em}
\renewcommand\labelenumi{(\theenumi)}
\begin{enumerate}
\item $K^{G_{\dimvec{d}}}(\St_{\dimvec{d}})$ is a free $\Rpt(T_{\dimvec{d}})$-module of rank $\abdimvec{d}!$, with a basis corresponding to the $\OOcell$-cells of $\St_{\dimvec{d}}$;\label{item:A1}
\item After base change to fraction field, $\Kcurl^{G_{\dimvec{d}}}(\St_{\dimvec{d}})$ is a free $\Rptc(T_{\dimvec{d}})$-module of rank $\abdimvec{d}!$, with a new basis corresponding to the $T_{\dimvec{d}}$-fixed points of $\St_{\dimvec{d}}$;\label{item:A2}
\item As an $\Rpt(T_{\dimvec{d}})$-algebra, $K^{G_{\dimvec{d}}}(\St_{\dimvec{d}})$ is generated by the Demazure operators $\{D_i\}_{i=1}^{\abdimvec{d}-1}$;\label{item:A3}
\item $K^{G_{\dimvec{d}}}(\St_{\dimvec{d}})$ has a faithful action on $K^{G_{\dimvec{d}}}\left(\RRep_{\dimvec{d}}(Q)\right)$, which embeds $K^{G_{\dimvec{d}}}(\St_{\dimvec{d}})$ as a subalgebra of the endomorphism ring $\End_{\mathbb{Z}}\label{item:A4} \left(K^{G_{\dimvec{d}}}\left(\RRep_{\dimvec{d}}(Q)\right)\right)$. Also, we have an explicit formula for the Demazure operator action on $K^{G_{\dimvec{d}}}\left(\RRep_{\dimvec{d}}(Q)\right)$.
\item Any element in $K^{G_{\dimvec{d}}}(\St_{\dimvec{d}})$ can be written as formal sum of certain planar diagrams, and the algebraic structure of $K^{G_{\dimvec{d}}}(\St_{\dimvec{d}})$ can be understood in a diagrammatic way.\label{item:A5}
\end{enumerate} 
\endgroup
\end{theoremA}

We will show \eqref{item:A1} in Chapter \ref{chap:cellular_fibration_theorem} (see Table \ref{table:module_absolute}), \eqref{item:A2} in Chapter \ref{chap:localization} (see Theorem \ref{thm:localization_theorem}, Definition \ref{def:localization_basis}), \eqref{item:A3}, \eqref{item:A4} in Chapter \ref{chap:excess_intersection_formula} (see Proposition \ref{prop:generators} for \eqref{item:A3}, Proposition \ref{prop:faithfulness}, Theorem \ref{thm:Demazure_operator_1} for \eqref{item:A4}). \eqref{item:A5} will be explained in detail in Chapter \ref{chap:applications} (see Section \ref{sec:diagram}). Varieties as well as their stratifications and $T$-fixed points will be defined in Chapter \ref{chap:variety_stratification}, while the basic results on $K$-theory will be treated in Chapter \ref{chap:Ktheory}.

Similar methods should apply to the case of partial flag varieties and yield a $K$-theoretic version of the quiver Schur algebra introduced by Stroppel--Webster \cite{https://doi.org/10.48550/arxiv.1110.1115}. We also note that similar $K$-theoretic analogs in the context of Soergel bimodules were studied by Eberhardt in  \cite{Eberhardt2022Koszul,https://doi.org/10.48550/arxiv.2208.01665}. Using the work of Stroppel--Eberhardt \cite{https://doi.org/10.48550/arxiv.1110.1115}, one should be able to show that the derived categories of modules of the $K$-theoretic KLR algebra can be realized in terms of certain K-motives on $\RRep_{\dimvec{d}}(Q)$.

$\,$

We proceed as follows.

In Chapter \ref{chap:variety_stratification}, we fix notation and collect properties of quiver flag varieties. Especially, the Steinberg varieties are also defined as an incidence variety, and their properties are described for future use.

From Chapter \ref{chap:Ktheory} to Chapter \ref{chap:excess_intersection_formula}, we introduce general results of $K$-theory, and then specify them to our cases. Both $K$-theory and cohomology are defined in Chapter \ref{chap:Ktheory}, with examples and functorialities carefully discussed (in $K$-theoretical version). Three isomorphisms are also stated in $K$-theoretical version in Section \ref{sec:Thom_isomorphism}-\ref{sec:reduction}. We compute the module structure of $K$-groups by the cellular fibration theorem \ref{thm:cellular_fibration}, see Chapter \ref{chap:cellular_fibration_theorem}. For the computation of convolution product, we introduce another basis of $K$-groups (in the field of fractions) and compute the transition matrix by the localization formula \ref{thm:localization_formula}, see Chapter \ref{chap:localization}. Finally, we compute the convolution structure of $K$-theory (Proposition \ref{prop:convolution_product_formula} for $T_{\dimvec{d}}$-equivariant, and Theorem \ref{thm:Demazure_operator_1} for $G_{\dimvec{d}}$-equivariant) by the excess intersection formula \ref{thm:excess_intersection_formula}.

Different from the previous chapters, the three sections in Chapter \ref{chap:applications} are quite independent, and can be read in any order. In Section \ref{sec:generalization}, we slightly relax the conditions on quivers and group actions. Section \ref{sec:diagram} collect examples and present them by diagrams. In Section \ref{sec:AScompletion}, $K$-theory and cohomology are connected by the Atiyah--Segal completion theorem \ref{thm:Atiyah--Segal_completion_theorem}, and the Chern class and the Todd class emerge explicitly in examples.
\clearpage
\section*{Part II}


As explained in the introduction of Part \ref{part:Steinberg_varieties}, irreducible representations of convolution algebras, such as (quiver) Hecke algebras, can be realized in terms of Springer fibers or generalizations thereof.
While the geometry of Springer fibers can be very intricate, one often show that they admit affine pavings.

Affine pavings are an important concept in algebraic geometry similar to cellular decompositions in topology. A complex algebraic variety $X$ has an affine paving if $X$ has a filtration
$$0= X_0 \subset X_1 \subset \cdots \subset X_d=X$$
with $X_i$ closed and $X_{i+1} \setminus X_i$ isomorphic to some affine space $\mathbb{A}^k_{\mathbb{C}}$.

Affine pavings imply nice properties about the cohomology of varieties, for example the vanishing of cohomology in odd degrees. For other properties see \cite[1.7]{de1988homology}.

Affine pavings have been constructed in many cases, as for Grassmannians, flag varieties, as well as certain Springer fibers, quiver Grassmannians, and quiver flag varieties. Part \ref{part:partial_flag_varieties} focuses on the case of (strict) partial flag varieties which parameterize subrepresentations of a fixed indecomposable representation of a quiver. In particular, we consider quivers of Dynkin type or affine type.
In this case, affine pavings have been constructed in \cite{irelli2019cell} for quiver Grassmannians in all types and in \cite{maksimau2019flag} for partial flag varieties of type $A$ and $D$ (see Table \ref{table:result}). Besides, affine pavings 
have been constructed in \cite[Theorem 6.3]{eberhardt2022motivic} for strict partial flag varieties in type $\tilde{A}$ with cyclic orientation, which generalized the result in \cite{sauter2015cell} for complete quiver flag varieties in nilpotent representations of an oriented cycle. In this part, we will tackle the remaining cases.
\begin{theoremB}\label{thm:affine_paving}
Fix a quiver $Q$ and a representation $M$ of $Q$.
\begingroup
\setlist{itemsep=-0.4em}
\renewcommand\labelenumi{(\theenumi)}
\upshape
\begin{enumerate}
\item If $Q$ is Dynkin, then any (strict) partial flag variety $\Flag{}(M)$ has an affine paving;\label{item:B1}
\item If $Q$ is of type $\tilde{A}$ or $\tilde{D}$, then for any indecomposable representation $M$, the (strict) partial flag variety $\Flag{}(M)$ has an affine paving;\label{item:B2}
\item If $Q$ is of type $\tilde{E}$, assume that $\Flag{}(N)$ has an affine paving for any regular quasi-simple representation $N \in \rep(Q)$, then $\Flag{}(M)$ has an affine paving for any indecomposable representation $M$.\label{item:B3}
\end{enumerate}
\endgroup
\end{theoremB}

%The nocolor version 
\begingroup
\renewcommand{\arraystretch}{1.3}
\begin{table}[ht]
\centering
\vspace{0.5cm}
\begin{tabular}{|c|c|c|c|}
\hline
            & $\Grq(X)$                  & $\Flagd(X)$                          & $\Flagdstr(X)$          \\ \hline
$A$         & \multirow{3}{*}{\cite[Section 5]{irelli2019cell}} & \multirow{2}{*}{\cite[Theorem 2.20]{maksimau2019flag}}        & \multirow{2}{*}{Theorem \ref{thm:Dynkincase}}       \\ \cline{1-1}
$D$         &                            &                                      &                         \\ \cline{1-1} \cline{3-4} 
$E$         &                            & \multicolumn{2}{c|}{Theorem \ref{thm:Dynkincase}}                                          \\ \hline
$\tilde{A}$ & \multirow{3}{*}{\cite[Section 6]{irelli2019cell}} & \multicolumn{2}{c|}{\multirow{2}{*}{Theorem \ref{thm:affinecase}}}                         \\ \cline{1-1}
$\tilde{D}$ &                            & \multicolumn{2}{c|}{}                                          \\ \cline{1-1} \cline{3-4} 
$\tilde{E}$ &                            & \multicolumn{2}{c|}{reduced to the regular quasi-finite case.} \\ \hline
\end{tabular}
\vspace{1mm}
\caption{}\label{table:result}
\end{table}
\endgroup
We will show \eqref{item:B1} in Theorem \ref{thm:Dynkincase} and \eqref{item:B2}, \eqref{item:B3} in Theorem \ref{thm:affinecase}.

$\,$

We proceed as follows. In Chapter \ref{chap:flag=gr}, we discuss basic definitions and properties of partial flags. In Section \ref{sec:mainthm} we will prove key Theorems \ref{thm:main1} and \ref{thm:main2}, which allow us to construct affine pavings for quiver partial flag varieties inductively. We apply these theorems to partial flag varieties of Dynkin type, see Section \ref{sec:Dynkin}, and to partial flag varieties of affine type, see Section \ref{sec:affine}. We will combine and extend results from \cite{irelli2019cell} and \cite{maksimau2019flag}.
Following the arguments of \cite{maksimau2019flag} would require studying millions of cases when we  consider the Dynkin quivers of type $E$. To avoid this, we extend the methods of \cite{irelli2019cell} from quiver Grassmannian to quiver partial flag variety. This will reduce the case by case analysis to a feasible computation of (mostly) 8 critical cases, which we carry out in Section \ref{sec:Dynkin} and Section \ref{appendix:proofcomplement}. The reduction uses Auslander--Reiten theory which we recall in Section \ref{appendix:arth}.
%It's easy to see that $\Flag{1}(X)=\Grq(X)$. These geometrical objects can be divided into different pieces according to the dimension vectors of $M_1,\ldots,M_d$, and each piece have its own natural (complex/Zariski) topology. It was proved in \cite{irelli2019cell} that $\Grq(X)$ have an affine paving, and in \cite{maksimau2019flag} that $\Flagd(X)$ have the same property when $Q$ is Dynkin quiver of type $A/E$. Here we go one step further, the results are concluded in the Table \ref{table:result}.
%
%
%The idea of proof is very simple: first, we view the  quiver partial flag variety as the quiver Grassmannian of the more complicated quiver; then we establish the decomposition so that one may solve the problem by induction; finally we set a special way of decomposition for each indecomposable module so that we can avoid meeting the bad decomposition. These contents are in Section \ref{sec:flag=gr},\ref{sec:mainthm},\ref{sec:Dynkin}, accordingly.



%%%%%%%%%%%%%%%%%%%%%%%%%%%%%%%%%%%%%%%%%%%%%%%%%%%%%%%%%%%%%%%%%%%%%%%%%%%%%%%%%%%%%%%%%%%%%
\section*{Acknowledgement}
%\addcontentsline{toc}{section}{Acknowledgement}
First, I would like to thank my supervisor, Jens Niklas Eberhardt, for introducing me to these problems, discussing earlier drafts, as well as providing help with the write-up.  I’d also like to extend my gratitude to my supervisor Professor Dr. Catharina Stroppel, who provided me with support and suggestions. I would also like to thank Hans Franzen for answering some questions regarding \cite{irelli2019cell}, and thank Ruslan Maksimau, Francesco Esposito for some comments and suggestions. Finally, I am also grateful to Luozi Shi, Yilong Zhang and Haohao Liu for reading earlier drafts and giving me feedback. Thanks for everyone who encouraged and supported me in the last month of deadline chasing.
%My thanks go to Professor xyz for his / her guidance and advice in the preparation and finalization of this article
% I would like to express my gratitude to
%my advisor James Newton without whom this paper would never have been completed. I
%thank him for suggesting this problem and guiding me throughout the whole project. He
%spent many hours discussing earlier drafts with me and explained a lot of things to me.
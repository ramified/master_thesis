%%%%%%%%%%%%%%%%%%%%%%%%%%%%%%%%%%%%%%%%%%%%%%%%%%%%%%%%%%%%%%%%%%%%%%%%%%%%%%%%%%%%%%%%%%%%
\chapter{$K$-theory and cohomology theory}\label{chap:Ktheory}



From my humble point of view, there is no easy cohomology theory, in a sense that key properties are usually hard to prove. On the other hand, plenty of examples can be quickly computed once we grasp some properties and use them in black boxes. Therefore, we won't prove any properties we stated. We have no choice but to do so, for the restricted space and time.

The main reference for the $K$-theory is \cite[Chapter 5]{chriss1997representation}. 
{
%I don't know how to fix the problem where the latter $$$$ does not work.
\abovedisplayskip=1ex
  \belowdisplayskip=1ex
\begin{setting}
Throughout abstract results of $K$-theory, we use the following notations:
\begin{itemize}
\setlength\itemsep{-2mm}
\item $G$ stands for a linear algebraic group, i.e., a closed subgroup of $\GL_n(\mathbb{C})$.\footnote{The closed embedding $G \hookrightarrow \GL_n(\mathbb{C})$ is not considered as the data of $G$.} Denote
$m: G \times G \longrightarrow G$
as the multiplication map of $G$.
\item $X$ is a variety over $\mathbb{C}$, i.e., a reduced, separated scheme of finite type over $\mathbb{C}$. We assume $X$ to be quasi-projective.
\item 
Usually, $X$ is equipped with an algebraic $G$-action (which is compatible with the variety structure of $G$ and $X$), then we say that $X$ is a $G$-variety. In that case, we will denote 
$\alpha: G \times X \longrightarrow X$
as the $G$-action map.
\item $\mathcal{F}$ is usually a sheaf on $X$, which is not flag variety $\GL_n/B$.
\end{itemize}
\end{setting}
}

\section{Definitions and initial examples}
\subsection{$G$-equivariant sheaf and $K^G_0(X)$}
We give definition for $K$-theory, which is lengthy already. Roughly speaking, a $G$-equivariant coherent sheaf over $X$ is a sheaf $\mathcal{F} \in \Coh(X)$ equipped with $G$-action which is compatible with the $G$-action on $X$, and $K$-theory is the Grothendieck group of $G$-equivariant coherent sheaves over $X$.

\begin{defn}[{G-equivariant sheaf, \cite[Definition 5.1.6]{chriss1997representation}}]
For a $G$-variety $X$, denote $p_i^{jk}, p_i:=p_i^{123}, p_{ij}:=p_{ij}^{123}$ as projections onto some factors, as follows.\footnote{Be careful, under this convention, the projection map $p_3^{23}=p_3^{13}: G \times X \longrightarrow X$ has subscription $3$, and $p_2$ means the projection from $G \times G \times X$ to the second $G$. This convention is different with notations in \cite[5.1]{chriss1997representation}.}
% https://q.uiver.app/?q=WzAsNyxbMiwwLCJHIFxcdGltZXMgRyBcXHRpbWVzIFgiXSxbMSwxLCJHIFxcdGltZXMgRyJdLFszLDEsIkcgXFx0aW1lcyBYIl0sWzUsMSwiRyBcXHRpbWVzIFgiXSxbNCwyLCJYIl0sWzIsMiwiRyJdLFswLDIsIkciXSxbMSw2LCJwXzFeezEyfSIsMl0sWzAsMSwicF97MTJ9IiwyXSxbMCwyLCJwX3syM30iXSxbMiw0LCJwXzNeezIzfSJdLFsxLDUsInBfMl57MTJ9Il0sWzIsNSwicF8yXnsyM30iLDJdLFszLDQsInBfM157MTN9IiwwLHsibGFiZWxfcG9zaXRpb24iOjMwLCJzdHlsZSI6eyJib2R5Ijp7Im5hbWUiOiJkYXNoZWQifX19XSxbMyw2LCJwXzFeezEzfSIsMix7ImxhYmVsX3Bvc2l0aW9uIjoyMCwic3R5bGUiOnsiYm9keSI6eyJuYW1lIjoiZGFzaGVkIn19fV0sWzAsMywicF97MTN9IiwwLHsiY3VydmUiOi0yfV1d
\[\begin{tikzcd}[column sep={1.5cm,between origins}]
	&& {G \times G \times X} &&&[2cm]\\
	& {G \times G} && {G \times X} && {G \times X} \\
	G && G && X &
	\arrow["{p_1^{12}}"', from=2-2, to=3-1]
	\arrow["{p_{12}}"', from=1-3, to=2-2]
	\arrow["{p_{23}}", from=1-3, to=2-4]
	\arrow["{p_3^{23}}", from=2-4, to=3-5]
	\arrow["{p_2^{12}}", from=2-2, to=3-3]
	\arrow["{p_2^{23}}"', from=2-4, to=3-3]
	\arrow["{p_3^{13}}"{pos=0.3}, dashed, from=2-6, to=3-5]
	\arrow["{p_1^{13}}"'{pos=0.2}, dashed, from=2-6, to=3-1]
	\arrow["{p_{13}}", curve={height=-12pt}, from=1-3, to=2-6]
\end{tikzcd}\]
We have morphisms
% https://q.uiver.app/?q=WzAsMyxbMSwwLCJHIFxcdGltZXMgWCJdLFswLDAsIkcgXFx0aW1lcyBYIFxcdGltZXMgWCJdLFsyLDAsIlgiXSxbMCwyLCJcXGFscGhhIiwwLHsib2Zmc2V0IjotMn1dLFswLDIsInBfM157MjN9PXBfM157MTN9IiwwLHsib2Zmc2V0IjoyfV0sWzEsMCwiXFxJZF9HIFxcdGltZXMgXFxhbHBoYSJdLFsxLDAsIm0gXFx0aW1lcyBcXElkX1giLDAseyJvZmZzZXQiOi00fV0sWzEsMCwicF97MjN9IiwwLHsib2Zmc2V0Ijo0fV1d
\[\begin{tikzcd}[column sep=huge]
	{G \times G \times X} & {G \times X} & X
	\arrow["{p_3^{23}=p_3^{13}}", shift left=3, from=1-2, to=1-3]
	\arrow["\alpha", shift right=3, from=1-2, to=1-3]
	\arrow["{p_{23}}", from=1-1, to=1-2]
	\arrow["{m \times \Id_X}", shift left=6, from=1-1, to=1-2]
	\arrow["{\Id_G \times \alpha}", shift right=6, from=1-1, to=1-2]
\end{tikzcd}\]
which satisfies the "coequalizer conditions":
\begin{equation*}
\begin{aligned}
  p_3^{23} \circ (m \times \Id_X) =\;& p_3^{23} \circ p_{23}  & (g_1,g_2,x) &\longmapsto x\\ 
  p_3^{23} \circ (\Id_G \times \alpha) =\;& \alpha \circ p_{23}  & (g_1,g_2,x) &\longmapsto g_2x\\ 
  \alpha \circ (m \times \Id_X) =\;& \alpha \circ (\Id_G \times \alpha)  &\qquad (g_1,g_2,x) &\longmapsto g_1g_2x\\ 
\end{aligned}
\end{equation*}

A \textbf{$\mathbf{G}$-equivariant (coherent) sheaf} \footnote{we will omit the word "coherent" for shorter notation.} on $X$ is a sheaf $\mathcal{F} \in \Coh(X)$ equipped with an isomorphism
$$\phi_{\mathcal{F}}: p_3^{23,*}\mathcal{F} \longrightarrow \alpha^* \mathcal{F}$$
such that the following diagram commutes:
% https://q.uiver.app/?q=WzAsNixbMSwwLCIobSBcXHRpbWVzIFxcSWRfWCleKiBcXGFscGhhXiogXFxtYXRoY2Fse0Z9Il0sWzIsMCwiKG0gXFx0aW1lcyBcXElkX1gpXiogcF8zXnsyMywqfSBcXG1hdGhjYWx7Rn0iXSxbMCwxLCIoXFxJZF9HIFxcdGltZXMgXFxhbHBoYSleKiBcXGFscGhhXiogXFxtYXRoY2Fse0Z9Il0sWzMsMSwicF97MjN9XipwXzNeezIzLCp9IFxcbWF0aGNhbHtGfSJdLFsxLDIsIihcXElkX0cgXFx0aW1lcyBcXGFscGhhKV4qIHBfM157MjMsKn0gXFxtYXRoY2Fse0Z9Il0sWzIsMiwicF97MjN9XiogXFxhbHBoYV4qIFxcbWF0aGNhbHtGfSJdLFswLDEsIihtIFxcdGltZXMgXFxJZF9YKV4qIFxccGhpX3tcXG1hdGhjYWx7Rn19Il0sWzIsMCwiIiwwLHsic3R5bGUiOnsiaGVhZCI6eyJuYW1lIjoibm9uZSJ9fX1dLFsxLDMsIiIsMCx7Im9mZnNldCI6LTEsInN0eWxlIjp7ImhlYWQiOnsibmFtZSI6Im5vbmUifX19XSxbNCw1LCIiLDAseyJzdHlsZSI6eyJoZWFkIjp7Im5hbWUiOiJub25lIn19fV0sWzUsMywicF97MjN9XipcXHBoaV97XFxtYXRoY2Fse0Z9fSJdLFsyLDQsIihcXElkX0cgXFx0aW1lcyBcXGFscGhhKV4qIFxccGhpX3tcXG1hdGhjYWx7Rn19Il1d
\begin{equation}\label{eq:associative_constraint}
\begin{tikzcd}[column sep={2cm,between origins}]
	&{(m \times \Id_X)^* p_3^{23,*} \mathcal{F}}  &[3.5cm]{(m \times \Id_X)^* \alpha^* \mathcal{F}}  &\\
	{p_{23}^*p_3^{23,*} \mathcal{F}} &&&  {(\Id_G \times \alpha)^* \alpha^* \mathcal{F}}\\
	& {p_{23}^* \alpha^* \mathcal{F}} &{(\Id_G \times \alpha)^* p_3^{23,*} \mathcal{F}} 
	\arrow["{(m \times \Id_X)^* \phi_{\mathcal{F}}}", from=1-2, to=1-3]
	\arrow[no head,equal, from=2-1, to=1-2]
	\arrow[shift left=1, no head,equal, from=1-3, to=2-4]
	\arrow[no head,equal, from=3-2, to=3-3]
	\arrow["{(\Id_G \times \alpha)^* \phi_{\mathcal{F}}}", from=3-3, to=2-4]
	\arrow["{p_{23}^*\phi_{\mathcal{F}}}", from=2-1, to=3-2]
\end{tikzcd}
\end{equation}

A \textbf{($\mathbf{G}$-equivariant) morphism} $f: (\mathcal{F},\phi_{\mathcal{F}}) \longrightarrow (\mathcal{G},\phi_{\mathcal{G}})$ between two $G$-equivariant sheaves is a morphism $f:\mathcal{F} \longrightarrow \mathcal{G}$ in $\Coh(X)$ such that the diagram
% https://q.uiver.app/?q=WzAsNCxbMCwwLCJcXGFscGhhXiogXFxtYXRoY2Fse0Z9Il0sWzAsMSwiXFxhbHBoYV4qIFxcbWF0aGNhbHtHfSJdLFsxLDAsInBfM157MjMsKn0gXFxtYXRoY2Fse0Z9Il0sWzEsMSwicF8zXnsyMywqfSBcXG1hdGhjYWx7R30iXSxbMCwyLCJcXHBoaV97XFxtYXRoY2Fse0Z9fSJdLFsxLDMsIlxccGhpX3tcXG1hdGhjYWx7R319Il0sWzAsMSwiXFxhbHBoYV4qIGYiLDJdLFsyLDMsInBfM157MjMsKn0gZiJdXQ==
\begin{equation}\label{eq:equiv_morphism}
\begin{tikzcd}
	{ p_3^{23,*} \mathcal{F}} & {\alpha^* \mathcal{F}} \\
	{ p_3^{23,*} \mathcal{G}} & {\alpha^* \mathcal{G}}
	\arrow["{\phi_{\mathcal{F}}}", from=1-1, to=1-2]
	\arrow["{\phi_{\mathcal{G}}}", from=2-1, to=2-2]
	\arrow["{p_3^{23,*} f}"', from=1-1, to=2-1]
	\arrow["{\alpha^* f}", from=1-2, to=2-2]
\end{tikzcd}
\end{equation}
commutes.

We denote $\Coh^G(X)$ as the category of $G$-equivariant sheaves.


\end{defn}
\begin{defn}[$G$-equivariant $K$-theory]
For a $G$-variety $X$, the $G$-equivariant $K$-theory is defined as the Grothendieck group of $G$-equivariant coherent sheaves over $X$, i.e.,
$$K_0^G(X):= K_0(\Coh^G(X)).$$
Specifically, for a point $\pt=\Spec \mathbb{C}$ with trivial $G$-action, denote
$$\Rpt(G):= K_0^G(\pt)=K_0(\Rep(G))$$
as the representation ring of group $G$.

We may omit $0$ for the convenience of writing and typing.
\end{defn}

Let us unravel this construction a little bit. For (geometrical) points $g,g_1,g_2 \in G$, denote that
\begin{equation*}
\begin{aligned}
  \iota_{g}:\;&  X \longrightarrow G \times X \qquad & x & \longmapsto (g,x)  \\ 
  \iota_{g_1,g_2}:\;&  X \longrightarrow G \times G \times X \qquad & x & \longmapsto (g_1,g_2,x)  \\
  \alpha_{g}:\;& \hspace{-2mm} \begin{tikzcd}[column sep=7mm]
  	X & {G \times X} & X
  	\arrow["\alpha", from=1-2, to=1-3]
  	\arrow["{\iota_g}", hook, from=1-1, to=1-2]
  \end{tikzcd} \qquad & x & \longmapsto gx  \\
\end{aligned}
\end{equation*}
By pulling back along $\iota_{g}$ and $\iota_{g_1,g_2}$, we can see geometrical meanings in the expressions. Apply $\iota_{g}^{*}$ to $\phi_{\mathcal{F}}$, one get
% https://q.uiver.app/?q=WzAsNCxbMiwwLCIgXFxwaGlfe2cseH1ee1xcbWF0aGNhbHtGfX1cXGhhdHs9fVxcbGVmdChcXGlvdGFfe2d9XnsqfVxccGhpX3tcXG1hdGhjYWx7Rn19XFxyaWdodClfeDogXFxtYXRoY2Fse0Z9X3giXSxbMCwwLCJcXGlvdGFfe2d9XnsqfVxccGhpX3tcXG1hdGhjYWx7Rn19OiBcXG1hdGhjYWx7Rn0iXSxbMSwwLCJcXGFscGhhX3tnfV57Kn1cXG1hdGhjYWx7Rn0iXSxbMywwLCJcXG1hdGhjYWx7Rn1fe2d4fSJdLFswLDNdLFsxLDJdLFsyLDAsIiIsMCx7InNob3J0ZW4iOnsic291cmNlIjozMCwidGFyZ2V0IjozMH0sInN0eWxlIjp7ImJvZHkiOnsibmFtZSI6InNxdWlnZ2x5In19fV1d
\[\begin{tikzcd}
	{\iota_{g}^{*}\phi_{\mathcal{F}}: \mathcal{F}} & {\alpha_{g}^{*}\mathcal{F}} &[1.5cm] { \phi_{g,x}^{\mathcal{F}}\hat{=}\left(\iota_{g}^{*}\phi_{\mathcal{F}}\right)_x: \mathcal{F}_x} & {\mathcal{F}_{gx}}
	\arrow[from=1-3, to=1-4]
	\arrow[from=1-1, to=1-2]
	\arrow[shorten <=7mm, shorten >=7mm,squiggly={
	               pre length=7mm, post length=7mm
	             }, maps to, from=1-2, to=1-3]
\end{tikzcd}\]
Therefore, $\phi_{\mathcal{F}}$ encodes information of $G$-action on $\mathcal{F}$, which is $G$-equivariant.

Now we apply $\iota_{g_1,g_2}^{*}$ to \eqref{eq:associative_constraint}:
% https://q.uiver.app/?q=WzAsOCxbMywwLCIgXFxtYXRoY2Fse0Z9X3giXSxbMCwwLCJcXG1hdGhjYWx7Rn0iXSxbMiwwLCJcXGFscGhhX3tnXzEsZ18yfV57Kn1cXG1hdGhjYWx7Rn09XFxhbHBoYV97Z18xfV57Kn1cXGFscGhhX3tnXzJ9XnsqfVxcbWF0aGNhbHtGfSJdLFs1LDAsIlxcbWF0aGNhbHtGfV97Z18xZ18yeH0iXSxbMSwyLCJcXGFscGhhX3tnXzJ9XnsqfVxcbWF0aGNhbHtGfSJdLFs0LDIsIlxcbWF0aGNhbHtGfV97Z18yeH0iXSxbMiwxXSxbMywxXSxbMCwzLCJcXHBoaV97Z18xZ18yLHh9XntcXG1hdGhjYWx7Rn19Il0sWzEsMiwiXFxpb3RhX3tnXzEsZ18yfV57Kn1cXHBoaV97XFxtYXRoY2Fse0Z9fSJdLFsxLDQsIlxcaW90YV97Z18yfV57Kn1cXHBoaV97XFxtYXRoY2Fse0Z9fSIsMl0sWzQsMiwiXFxpb3RhX3tnXzF9XnsqfVxccGhpX3tcXGFscGhhX3tnXzJ9XnsqfVxcbWF0aGNhbHtGfX0iLDJdLFswLDUsIlxccGhpX3tnXzIseH1ee1xcbWF0aGNhbHtGfX0iLDJdLFs1LDMsIlxccGhpX3tnXzEsZ18yeH1ee1xcbWF0aGNhbHtGfX09XFxwaGlfe2dfMSx4fV57XFxhbHBoYV97Z18yfV57Kn1cXG1hdGhjYWx7Rn19IiwyXSxbNiw3LCIiLDAseyJzaG9ydGVuIjp7InNvdXJjZSI6MzAsInRhcmdldCI6MzB9LCJzdHlsZSI6eyJib2R5Ijp7Im5hbWUiOiJzcXVpZ2dseSJ9fX1dXQ==
\[\begin{tikzcd}[column sep={2cm,between origins}, row sep=small]
	{\mathcal{F}} && {\makebox[10ex][l]{$\alpha_{g_1,g_2}^{*}\mathcal{F}=\alpha_{g_1}^{*}\alpha_{g_2}^{*}\mathcal{F}$}} & [2cm]{ \mathcal{F}_x} && {\mathcal{F}_{g_1 g_2 x}} \\
	&& {} & {} \\
	& {\alpha_{g_2}^{*}\mathcal{F}} &&& {\mathcal{F}_{g_2x}}
	\arrow["{\phi_{g_1g_2,x}^{\mathcal{F}}}", from=1-4, to=1-6]
	\arrow["{\iota_{g_1,g_2}^{*}\phi_{\mathcal{F}}}", from=1-1, to=1-3]
	\arrow["{\iota_{g_2}^{*}\phi_{\mathcal{F}}}"', from=1-1, to=3-2]
	\arrow["{\iota_{g_1}^{*}\phi_{\alpha_{g_2}^{*}\mathcal{F}}}"', from=3-2, to=1-3]
	\arrow["{\phi_{g_2,x}^{\mathcal{F}}}"', from=1-4, to=3-5]
	\arrow["{\phi_{g_1,g_2x}^{\mathcal{F}}=\phi_{g_1,x}^{\alpha_{g_2}^{*}\mathcal{F}}}"', from=3-5, to=1-6]
	\arrow[shorten <=17mm, shorten >=7mm,squiggly={
		               pre length=17mm, post length=7mm
		             }, maps to, from=2-3, to=2-4]
\end{tikzcd}\]
So \eqref{eq:associative_constraint} is just the associative constraint of the $G$-structure on $\mathcal{F}$.

Similarly, apply $\iota_{g}^{*}$ to \eqref{eq:equiv_morphism}, we get
% https://q.uiver.app/?q=WzAsMTAsWzIsMCwiIFxcbWF0aGNhbHtGfV94Il0sWzAsMCwiXFxtYXRoY2Fse0Z9Il0sWzEsMCwiXFxhbHBoYV97Z31eeyp9XFxtYXRoY2Fse0Z9Il0sWzMsMCwiXFxtYXRoY2Fse0Z9X3tneH0iXSxbMCwyLCJcXG1hdGhjYWx7R30iXSxbMiwyLCIgXFxtYXRoY2Fse0d9X3giXSxbMSwxXSxbMiwxXSxbMSwyLCJcXGFscGhhX3tnfV57Kn1cXG1hdGhjYWx7R30iXSxbMywyLCJcXG1hdGhjYWx7R31fe2d4fSJdLFswLDMsIlxccGhpX3tnLHh9XntcXG1hdGhjYWx7Rn19Il0sWzEsMiwiXFxpb3RhX3tnfV57Kn1cXHBoaV97XFxtYXRoY2Fse0Z9fSJdLFsxLDQsImYiLDJdLFswLDUsImZfeCIsMl0sWzYsNywiIiwwLHsic2hvcnRlbiI6eyJzb3VyY2UiOjMwLCJ0YXJnZXQiOjMwfSwic3R5bGUiOnsiYm9keSI6eyJuYW1lIjoic3F1aWdnbHkifX19XSxbMiw4LCJcXGFscGhhX3tnfV57Kn1mIl0sWzQsOCwiXFxpb3RhX3tnfV57Kn1cXHBoaV97XFxtYXRoY2Fse0d9fSJdLFs1LDksIlxccGhpX3tnLHh9XntcXG1hdGhjYWx7R319Il0sWzMsOSwiZl97Z3h9Il1d
\[\begin{tikzcd}[row sep=tiny]
	{\mathcal{F}} & {\alpha_{g}^{*}\mathcal{F}} &[15mm] { \mathcal{F}_x} & {\mathcal{F}_{gx}} \\
	& {} & {} \\
	{\mathcal{G}} & {\alpha_{g}^{*}\mathcal{G}} & { \mathcal{G}_x} & {\mathcal{G}_{gx}}
	\arrow["{\phi_{g,x}^{\mathcal{F}}}", from=1-3, to=1-4]
	\arrow["{\iota_{g}^{*}\phi_{\mathcal{F}}}", from=1-1, to=1-2]
	\arrow["f"', from=1-1, to=3-1]
	\arrow["{f_x}"', from=1-3, to=3-3]
	\arrow[shorten <=9mm, shorten >=9mm,squiggly={
		               pre length=9mm, post length=9mm
		             }, maps to, from=2-2, to=2-3]
	\arrow["{\alpha_{g}^{*}f}", from=1-2, to=3-2]
	\arrow["{\iota_{g}^{*}\phi_{\mathcal{G}}}", from=3-1, to=3-2]
	\arrow["{\phi_{g,x}^{\mathcal{G}}}", from=3-3, to=3-4]
	\arrow["{f_{gx}}", from=1-4, to=3-4]
\end{tikzcd}\]
So \eqref{eq:equiv_morphism} is just the condition for $f$ to be $G$-equivariant.

There are two extreme situations worth mentioning about. 
When $G=\Id$, there is no $G$-action structure constrain on varieties and sheaves. Therefore,
$$\Coh^{\Id}(X)=\Coh(X) \qquad K_0^{\Id}(X)=K_0(X) \,\hat{=}\, K_0(\Coh(X)).$$
When $G$ acts on $X=\Spec A$ trivially, any sheaf $\mathcal{F} \in \Coh^G(X)$ can be viewed as an (finitely generated)\footnote{We already assume $X$ to be of finite type, so coherent condition is equivalent to finitely generated condition.} $A$-module $M$ with $G$-action, so
$$\Coh^G(X)=\rep_A(G) \xlongequal{\text{when $G$ is finite}} \Mod(A[G]).$$
In particular, any sheaf $\mathcal{F} \in \Coh^G(\pt)$ can be viewed as a finite dimensional complex $G$-representation, so
$$\Coh^G(\pt)=\rep_{\mathbb{C}}(G) \xlongequal{\text{when $G$ is finite}} \Mod(\mathbb{C}[G]).$$

???(If I have time I will compute $K_0(\mathbb{P}^1)$ here.)
\subsection{Representation ring $\Rpt(G)$}\label{subset:rep_ring}
Now let us try to figure out some examples.

Recall that any coherent sheaf over $\pt$ is equivalent to a finite dimensional $\mathbb{C}$-vector space, and any $G$-equivariant coherent sheaf over $\pt$ is equivalent to a finite dimensional complex $G$-representation. Moreover, by Jordan-Hölder theorem, every finite dimensional complex $G$-representation can be written as a composition series such that each quotient object is irreducible. Therefore,
$$\Rpt(G)=\bigoplus_{\rho \in \Irr(G)} \mathbb{Z}$$
as a free $\mathbb{Z}$-module.

For $\Rpt(G)$, we have the multiplication structure induced by tensor products on complex $G$-representations. Let us see some examples now. We use Setting \ref{set:initial_case} in these examples.

\begin{eg}\label{eg:K-initial-1}
For trivial group $\Id$, every $\Id$-representation is just a $\mathbb{C}$-vector space, which can be written as the direct sum of $1$-dimensional vector spaces. Therefore,
$$\Rpt(\Id)=\mathbb{Z}.$$
\end{eg}

\begin{eg}\label{eg:K-initial-2}
For group $T$, since $T$ is abelian, every $T$-representation can be written as direct sum of $1$-dimensional vector spaces. Furthermore,
\begin{equation*}
\begin{aligned}
  \Irr(T)=\;&  \left\{ \rho: T \longrightarrow \mathbb{C}^{\times} \;\middle|\;\rho \text{ is an (algebraic) group homomorphism}  \right\} \\ 
  =\;& \Hom_{\mathbb{C}\Alggp}(T, \mathbb{C}^{\times}):= X^{*}(T)
\end{aligned}
\end{equation*} 
We get
$$\Rpt(T)=\bigoplus_{\rho \in \Irr(T)} \mathbb{Z}=\mathbb{Z}\left[X^{*}(T) \right].$$
The group structure in $X^{*}(T)$ is given by tensor product, so the multiplication structure is induced by the group structure in $X^{*}(T)$. Denote 
$$\varepsilon_i: T \longrightarrow \mathbb{C}^{\times} \qquad \begin{pmatrix}
t_1 &&&&\\[-3mm]
& \ddots&&&\\[-1mm]
&&t_i&&\\[-3mm]
&&& \ddots&\\[-1mm]
&&&&t_n\\
\end{pmatrix} \longmapsto t_i$$
as a $\mathbb{Z}$-basis of $X^{*}(T)$, then $X^*(T) \cong \oplus_{i=1}^{n}\mathbb{Z} \varepsilon_i$.

To distinguish the addition in $X^{*}(T)$ and $\mathbb{Z}\left[ X^{*}(T) \right]$, we rewrite $\varepsilon_i$ as $e_i$. In that case, $\sum_{i=1}^{n}k_i\varepsilon_i$ is sent to $\prod_{i=1}^{n}e_{i}^{k_i}$, and 

$$\Rpt(T) \cong \mathbb{Z}\!\left[ e_1^{\pm 1},\ldots,e_n^{\pm 1} \right]$$
as a $\mathbb{Z}$-algebra.

By forgetting $T$-actions, we get a morphism of $\mathbb{Z}$-algebra
$$\Rpt(T) \longrightarrow \Rpt(\Id) \qquad f(e_1,\ldots,e_n) \longmapsto f(1,\ldots,1).$$
\end{eg}

\begin{eg}\label{eg:K-initial-3}
After we state the reduction isomorphism \ref{prop:reduction_isomorphism}, we can show that
$$\Rpt(N) \cong \Rpt(\Id) \cong \mathbb{Z} \qquad \Rpt(B) \cong \Rpt(T) \cong \mathbb{Z}\!\left[ e_1^{\pm 1},\ldots,e_n^{\pm 1} \right]$$
\end{eg}

\begin{eg}\label{eg:K-initial-4}
By \cite[Theorem 6.1.4]{chriss1997representation},
$$\Rpt(\GL_n) \cong \Rpt(T)^{W} \cong \mathbb{Z}\!\left[ e_1^{\pm 1},\ldots,e_n^{\pm 1} \right]^{S_n}.$$
This can be viewed as a "group" analogue of Chevalley restriction theorem. Notice that we have clear description of finite dimensional irreducible representations of $\GL_n$, and the forget map
% https://q.uiver.app/?q=WzAsNCxbMCwwLCJcXHJlcChcXEdMX24pIl0sWzEsMCwiXFxyZXAoVCkiXSxbMiwwLCJcXFJwdChcXEdMX24pIl0sWzMsMCwiXFxScHQoVCkiXSxbMiwzXSxbMCwxXSxbMSwyLCIiLDAseyJzdHlsZSI6eyJ0YWlsIjp7Im5hbWUiOiJtYXBzIHRvIn0sImJvZHkiOnsibmFtZSI6InNxdWlnZ2x5In19fV1d
\[\begin{tikzcd}
	{\rep(\GL_n)} & {\rep(T)} &[1.5cm] {\Rpt(\GL_n)} & {\Rpt(T)}
	\arrow[from=1-3, to=1-4]
	\arrow[from=1-1, to=1-2]
	\arrow[shorten <=7mm, shorten >=7mm,squiggly={
		               pre length=7mm, post length=7mm
		             }, maps to, from=1-2, to=1-3]
\end{tikzcd}\]
views $\GL_n$-representations as special $W$-invariant $T$-representations.
\end{eg}

From these examples we already see the difficulty of computing $K$-theories. Therefore, a series of properties of $K$-theories are definitely needed for computations. To state these properties, we need to define some tools (or weapons???) in $K$-theory.
\section[Three functors]{Three functors: pullback, proper pushforward and tensor product}
In this section, we will construct three basic functors of equivariant $K$-theory: pullback, proper pushforward and tensor product.

\subsection{Non-derived three functors in $\Coh^G(X)$} 
We assume that readers know the non-derived pullback, pushforward and tensor product of normal coherent sheaves. (See \cite[Chapter 16]{vakil2017rising})

As a special reminder, the pushforward of coherent sheaves may be not coherent. This problem can be remedied by Grothendieck's coherence theorem \cite[Theorem 18.9.1]{vakil2017rising}, once we impose morphisms to be proper (and Noetherian hypotheses on varieties). That is why we only consider about proper pushforward.

Now let us consider the effect of $G$-equivariance. Somewhat surprising, these three functors behave quite well with group actions.



\begin{defn}[Group action on pullback, proper pushforward and tensor product]
Let $X,Y$ be $G$-varieties, $f: Y \longrightarrow X$ be a $G$-equivariant morphism. For $(\mathcal{G},\phi_{\mathcal{G}}) \in \Coh^G(Y)$, we define group actions on $f^*\mathcal{F}$, $f_*\mathcal{G}$ and $\mathcal{F} \otimes \mathcal{G}$, as follows.

% https://q.uiver.app/?q=WzAsMTAsWzAsMywiRyBcXHRpbWVzIFgiXSxbMSwzLCJYIl0sWzAsMSwiRyBcXHRpbWVzIFkiXSxbMSwxLCJZIl0sWzIsMiwiXFxtYXRoY2Fse0Z9Il0sWzIsMCwiXFxtYXRoY2Fse0d9Il0sWzMsMywiRyBcXHRpbWVzIFgiXSxbNCwzLCJYIl0sWzUsMiwiXFxtYXRoY2Fse0Z9Il0sWzYsMiwiXFxtYXRoY2Fse0Z9JyJdLFswLDEsInBfezMsWH1eezIzfSIsMCx7Im9mZnNldCI6LTJ9XSxbMCwxLCJcXGFscGhhX1giLDAseyJvZmZzZXQiOjJ9XSxbMiwzLCJwX3szLFl9XnsyM30iLDAseyJvZmZzZXQiOi0yfV0sWzIsMywiXFxhbHBoYV9ZIiwyLHsib2Zmc2V0IjoyfV0sWzIsMCwiXFxJZF9HIFxcdGltZXMgZiIsMl0sWzMsMSwiZiJdLFsyLDEsIiIsMix7InN0eWxlIjp7Im5hbWUiOiJjb3JuZXIifX1dLFs1LDMsIiIsMix7InN0eWxlIjp7ImhlYWQiOnsibmFtZSI6Im5vbmUifX19XSxbNCwxLCIiLDIseyJzdHlsZSI6eyJoZWFkIjp7Im5hbWUiOiJub25lIn19fV0sWzYsNywicF97MyxYfV57MjN9IiwwLHsib2Zmc2V0IjotMn1dLFs2LDcsIlxcYWxwaGFfWCIsMCx7Im9mZnNldCI6Mn1dLFs4LDcsIiIsMix7InN0eWxlIjp7ImhlYWQiOnsibmFtZSI6Im5vbmUifX19XSxbOSw3LCIiLDIseyJzdHlsZSI6eyJoZWFkIjp7Im5hbWUiOiJub25lIn19fV1d
\[\begin{tikzcd}[column sep={between origins, 25mm}]
	&&[-20mm] {\mathcal{G}}&&& [-30mm]&[-15mm]\\[-4mm]
	{G \times Y} & Y \\
	&& {\mathcal{F}} &&& {\mathcal{F}} & {\mathcal{F}'} \\[-4mm]
	{G \times X} & X && {G \times X} & X
	\arrow["{p_{3,X}^{23}}", shift left=2, from=4-1, to=4-2]
	\arrow["{\alpha_X}", shift right=2, from=4-1, to=4-2]
	\arrow["{p_{3,Y}^{23}}", shift left=2, from=2-1, to=2-2]
	\arrow["{\alpha_Y}"', shift right=2, from=2-1, to=2-2]
	\arrow["{\Id_G \times f}"', from=2-1, to=4-1]
	\arrow["f"', from=2-2, to=4-2]
	\arrow["\lrcorner"{anchor=center, pos=0.125}, draw=none, from=2-1, to=4-2]
	\arrow[no head, from=1-3, to=2-2]
	\arrow[no head, from=3-3, to=4-2]
	\arrow["{p_{3,X}^{23}}", shift left=2, from=4-4, to=4-5]
	\arrow["{\alpha_X}", shift right=2, from=4-4, to=4-5]
	\arrow[no head, from=3-6, to=4-5]
	\arrow[no head, from=3-7, to=4-5]
\end{tikzcd}\]
By definition, we get $$p_{3,X}^{23} \circ \left(\Id_G \times f\right) = f \circ p_{3,Y}^{23}.$$ 
Since $f$ is $G$-equivariant, $$\alpha_X \circ \left(\Id_G \times f\right) = f \circ \alpha_Y.$$
These two diagrams are Cartesian, and $p_{3,X}^{23}, \alpha_X$ are flat.

The pullback $(f^*\mathcal{F},\phi_{f^*\mathcal{F}}) \in \Coh^G(Y)$ is defined by
$$\phi_{f^* \mathcal{F}}: p_{3,Y}^{23,*}f^*\mathcal{F}=\left( \Id_G \times f  \right)^* p_{3,X}^{23,*}\mathcal{F} 
\begin{tikzcd}[column sep=20mm]
	{} & {}
	\arrow["{\left( \Id_G \times f  \right)^* \phi_{\mathcal{F}}}", from=1-1, to=1-2]
\end{tikzcd} 
\left( \Id_G \times f  \right)^* \alpha_{X}^*\mathcal{F} =\alpha_{Y}^* f^*\mathcal{F}$$

By flat base change \cite[Theorem 24.2.8]{vakil2017rising}, assuming $f$ is proper, the proper pushforward $(f_{*}\mathcal{G},\phi_{f_{*}\mathcal{G}}) \in \Coh^G(X)$ is defined by
$$\phi_{f_{*}\mathcal{G}}: p_{3,X}^{23,*}f_{*}\mathcal{G}\cong\left( \Id_G \times f  \right)_{*} p_{3,Y}^{23,*}\mathcal{G} 
\begin{tikzcd}[column sep=20mm]
	{} & {}
	\arrow["{\left( \Id_G \times f  \right)_{*} \phi_{\mathcal{G}}}", from=1-1, to=1-2]
\end{tikzcd}
 \left( \Id_G \times f  \right)_{*} \alpha_{Y}^*\mathcal{G} \cong\alpha_{X}^*f_{*}\mathcal{G}$$

In general, we can also define $(\Ri f_{*}\mathcal{G},\phi_{\Ri f_{*}\mathcal{G}}) \in \Coh^G(X)$ by
$$\phi_{\Ri f_{*}\mathcal{G}}: p_{3,X}^{23,*}\Ri f_{*}\mathcal{G}\cong\Ri \left( \Id_G \times f  \right)_{*} p_{3,Y}^{23,*}\mathcal{G} 
\begin{tikzcd}[column sep=20mm]
	{} & {}
	\arrow["{\Ri\, \left( \Id_G \times f  \right)_{*} \phi_{\mathcal{G}}}", from=1-1, to=1-2]
\end{tikzcd}
 \Ri \left( \Id_G \times f  \right)_{*} \alpha_{Y}^*\mathcal{G} \cong\alpha_{X}^*\Ri f_{*}\mathcal{G}$$
 
 Similarly, the tensor product $(\mathcal{F} \otimes \mathcal{F}', \phi_{\mathcal{F} \otimes \mathcal{F}'}) \in \Coh^G(X)$ is defined by
 $$\phi_{\mathcal{F} \otimes \mathcal{F}'}: p_{3,X}^{23,*}\left(\mathcal{F} \otimes \mathcal{F}'\right) \cong p_{3,X}^{23,*} \mathcal{F} \otimes p_{3,X}^{23,*} \mathcal{F}'
 \begin{tikzcd}[column sep=20mm]
 	{} & {}
 	\arrow["{ \phi_{\mathcal{F}} \otimes \phi_{\mathcal{F}'}}", from=1-1, to=1-2]
 \end{tikzcd} 
 \alpha_{X}^* \mathcal{F} \otimes \alpha_{X}^* \mathcal{F}' \cong\alpha_{X}^* \left(\mathcal{F} \otimes \mathcal{F}'\right).$$
\end{defn}

The following definition will be useful in redefining tensor products.
\begin{defn}[External tensor product]
For two $G$-varieties $X$ and $Y$, define a functor 
$$\boxtimes: \Coh^G(X) \times \Coh^G(Y) \longrightarrow \Coh^G(X \times Y) \qquad (\mathcal{F},\mathcal{G}) \longmapsto \mathcal{F} \boxtimes \mathcal{G}$$
where
$$\mathcal{F} \boxtimes \mathcal{G}:=p_X^*\mathcal{F} \otimes p_X^*\mathcal{G}.$$
$\boxtimes$ is called the \textbf{external tensor product}.
\end{defn}
\begin{remark}\label{rmk:ext_tensor_product}
For $G$-variety $X$ and $\mathcal{F}$, $\mathcal{F}' \in \Coh^G(X)$, denote $\Delta: X \hookrightarrow X \times X$ to be the diagonal embedding, we have
$$\mathcal{F} \otimes \mathcal{F}' \cong \Delta^* (\mathcal{F} \boxtimes \mathcal{F}')$$
Unlike $\otimes$, $\boxtimes$ is always an exact functor. This feature let us redefine tensor product in $K$-theory later on.
\end{remark}
\subsection{Smooth case}
We would like to extend functors in $\Coh^G(X)$ to $K^G(X)$. However, these (non-derived) functors are usually not exact, so we have to work over ($G$-equivariant) derived category of cohrent sheaves $\Dcoh^{G}(X)$ and replace every functor by its derived version.

Still, we can not extend functors from $\Dcoh^G(X)$ to $K^G(X)$. The chain complex in $\Dcoh^G(X)$ can have infinite many non-zero terms, which can not be viewed as an element in $K^G(X)$. Therefore, we consider the bounded ($G$-equivariant) derived category $\Dcoh^{b,G}(X)$ as a full subcategory of $\Dcoh^G(X)$.

The last problem comes when we restrict functors to $\Dcoh^{b,G}(X)$:
\begin{equation*}
\begin{aligned}
  f^*:\;& \Dcoh^{b,G}(X) \longrightarrow \Dcoh^{G}(Y) \\
  f_*:\;& \Dcoh^{b,G}(Y) \longrightarrow \Dcoh^{b,G}(X) \\
  \otimes:\;& \Dcoh^{b,G}(X) \times \Dcoh^{b,G}(X) \longrightarrow \Dcoh^{G}(X) \\   
\end{aligned}
\end{equation*}
Other than proper pushforward, \footnote{See \cite[5.2.13]{chriss1997representation} for proper pushforward preserving boundness, and it essentially use the higher cohomology vanishing theorem \cite[Theorem 18.8.5]{vakil2017rising}.} pullback and tensor product may not preserve boundness. 


For pullback, boundness preserving is equivalent to the following condition:
\begin{equation}\label{eq:assumption0}
\text{$f: Y \longrightarrow X$ is $G$-equivariant of globally finite $\Tor$-dimension.}
\end{equation}

When $X,Y$ are smooth, the condition \eqref{eq:assumption0} is automatically satisfied. (See \cite[5.2.5(ii)]{chriss1997representation}). The condition is concluded as follows:
\begin{equation}\label{eq:assumption1}
\text{$X$, $Y$ are smooth $G$-varieties, and $f: Y \longrightarrow X$ is $G$-equivariant.}
\end{equation}

Tensor product also preserve boundness when $X$ is smooth. By Remark \ref{rmk:ext_tensor_product}, $\boxtimes$ is exact, and $\Delta^*$ preserves boundness when $X$ is smooth, so $\otimes$ also preserves boundness. In particular, one can define tensor product on $K^G(X)$ for $X$ smooth:
$$\otimes: K^G(X) \times K^G(X) 
 \begin{tikzcd}[column sep=5mm]
 	{} & {}
 	\arrow["{\boxtimes}", from=1-1, to=1-2]
 \end{tikzcd} 
 K^G(X \times X) 
 \begin{tikzcd}[column sep=5mm]
 	{} & {}
 	\arrow["{\Delta^*}", from=1-1, to=1-2]
 \end{tikzcd}  
 K^G(X) \qquad \mathcal{F} \otimes \mathcal{F}'=\Delta^*\left( \mathcal{F} \boxtimes \mathcal{F}' \right)
 $$

\begin{remark}
When $f: Y \longrightarrow X$ is open embedding, the non-derived pullback $f^*$ is exact, so we can define pullback on $K$-theory automatically.
\end{remark}
\subsection{Restriction with supports}
In practice, the varieties we consider are not smooth. Luckily, these varieties are always embedded in some ambiance spaces which are smooth.

\begin{defn}[Restriction with supports]
For a triple $(X,Y,f)$ satisfying assumption \eqref{eq:assumption1}, and a $G$-equivariant closed subvariety $Z$ of $X$, the triple $\left(Z,f^{-1}(X),f\big|_{f^{-1}(X)}  \right)$ is called a restriction with supports of $(X,Y,f)$.
\end{defn}
We can now define pullback of $f$ in the following assumption:
\begin{equation}\label{eq:assumption2}
\begin{aligned}
\text{$f:Y \longrightarrow X$ is $G$-equivariant, and $f$ is a restriction with supports}&\\
\text{by some $f':Y' \longrightarrow X'$, where $X'$, $Y'$ are smooth.}&
\end{aligned}
\end{equation}

\begin{defn}[Pullback with supports]
Let $Z,Z'$ be $G$-varieties, $h: Z' \longrightarrow Z$ be a $G$-equivariant closed embedding. Suppose that $h$ is a restriction with support of some $(X,Y,f)$ satisfying the assumption \eqref{eq:assumption1}, i.e., we have a $G$-equivariant closed embedding $\iota_Z: Z \longrightarrow X$ such that 
$Z' \cong f^{-1}(Z)$ and $h=f\big|_{Z'}$.
Denote $\iota_{Z'}: Z' \longrightarrow Y$ as the induced $G$-equivariant closed embedding, we would like to construct the pullback $h^*: K^G(Z) \longrightarrow K^G(Z')$.
\begin{equation}\label{eq:pullback_with_supports}
% https://q.uiver.app/?q=WzAsMTAsWzAsMCwiWiciXSxbMSwwLCJaIl0sWzAsMiwiWSJdLFsxLDIsIlgiXSxbMiwwLCJLXkcoWicpIl0sWzIsMiwiS15HKFkpIl0sWzMsMiwiS15HKFgpIl0sWzMsMCwiS15HKFopIl0sWzEsMV0sWzIsMV0sWzAsMSwiaCIsMCx7InN0eWxlIjp7InRhaWwiOnsibmFtZSI6Imhvb2siLCJzaWRlIjoidG9wIn19fV0sWzEsMywiXFxpb3RhX1oiLDAseyJzdHlsZSI6eyJ0YWlsIjp7Im5hbWUiOiJob29rIiwic2lkZSI6InRvcCJ9fX1dLFsyLDMsImYiXSxbMCwyLCJcXGlvdGFfe1onfSIsMix7InN0eWxlIjp7InRhaWwiOnsibmFtZSI6Imhvb2siLCJzaWRlIjoidG9wIn19fV0sWzQsNSwiXFxpb3RhX3taJywqfSIsMl0sWzYsNSwiZl4qIiwyXSxbNyw2LCJcXGlvdGFfe1osKn0iXSxbNSw0LCJcXGdyIiwyLHsiY3VydmUiOjIsInN0eWxlIjp7ImJvZHkiOnsibmFtZSI6ImRvdHRlZCJ9fX1dLFs3LDQsImheKiIsMix7InN0eWxlIjp7ImJvZHkiOnsibmFtZSI6ImRhc2hlZCJ9fX1dLFs4LDksIiIsMCx7InNob3J0ZW4iOnsic291cmNlIjoyMCwidGFyZ2V0IjoyMH0sInN0eWxlIjp7InRhaWwiOnsibmFtZSI6Im1hcHMgdG8ifSwiYm9keSI6eyJuYW1lIjoic3F1aWdnbHkifX19XV0=
\begin{tikzcd}[column sep={10mm}, row sep={4mm}]
	{Z'} & Z &[5mm] {K^G(Z')} & {K^G(Z)} \\
	& {} & {} \\
	Y & X & {K^G(Y)} & {K^G(X)}
	\arrow["h", hook, from=1-1, to=1-2]
	\arrow["{\iota_Z}", hook, from=1-2, to=3-2]
	\arrow["f", from=3-1, to=3-2]
	\arrow["{\iota_{Z'}}"', hook, from=1-1, to=3-1]
	\arrow["{\iota_{Z',*}}"', from=1-3, to=3-3]
	\arrow["{f^*}"', from=3-4, to=3-3]
	\arrow["{\iota_{Z,*}}", from=1-4, to=3-4]
	\arrow["\gr"', curve={height=12pt}, dotted, from=3-3, to=1-3]
	\arrow["{h^*}"', dashed, from=1-4, to=1-3]
	\arrow[shorten <=7mm, shorten >=9mm,squiggly={
			               pre length=7mm, post length=9mm
			             }, maps to, from=2-2, to=2-3]
\end{tikzcd}
\end{equation}


Follows \cite[5.2.7(ii)]{chriss1997representation}, one can construct a morphism
$$\gr: \Img \left(  f^* \circ \iota_{Z,*} \right) \longrightarrow K^G(Z'),$$
and the pullback is defined as
% https://q.uiver.app/?q=WzAsNCxbMywwLCJLXkcoWicpIl0sWzIsMCwiS15HKFkpIl0sWzEsMCwiS15HKFgpIl0sWzAsMCwiaF4qOkteRyhaKSJdLFsyLDEsImZeKiJdLFszLDIsIlxcaW90YV97WiwqfSJdLFsxLDAsIlxcZ3IiLDAseyJzdHlsZSI6eyJib2R5Ijp7Im5hbWUiOiJkb3R0ZWQifX19XV0=
\[\begin{tikzcd}
	{h^*:K^G(Z)} & {K^G(X)} & {K^G(Y)} & {K^G(Z').}
	\arrow["{f^*}", from=1-2, to=1-3]
	\arrow["{\iota_{Z,*}}", from=1-1, to=1-2]
	\arrow["\gr", dotted, from=1-3, to=1-4]
\end{tikzcd}\]
\end{defn}

\begin{warning}
The diagram \eqref{eq:pullback_with_supports} of $K$-group is usually not commutative. In fact, we will state the excess base change in Section \ref{sec:statement_localization}, in which the Euler class measures the failure of diagram to be commutative.
\end{warning}

\begin{defn}[Tensor product with supports/Intersection product]
Let $X$ be a smooth $G$-variety, and $Z$, $Z' \subseteq X$ be two closed $G$-subvarieties. The tensor product with supports is defined as
% https://q.uiver.app/?q=WzAsMyxbMiwwLCJLXkcoWiBcXGNhcCBaJykiXSxbMSwwLCJLXkcoWiBcXHRpbWVzIFonKSJdLFswLDAsIlxcb3RpbWVzOkteRyhaKSBcXHRpbWVzIEteRyhaJykiXSxbMSwwLCJcXERlbHRhXioiXSxbMiwxLCJcXGJveHRpbWVzIl1d
\[\begin{tikzcd}
	{\otimes:K^G(Z) \times K^G(Z')} & {K^G(Z \times Z')} & {K^G(Z \cap Z')}
	\arrow["{\Delta^*}", from=1-2, to=1-3]
	\arrow["\boxtimes", from=1-1, to=1-2]
\end{tikzcd}\]
i.e., $\mathcal{F} \otimes \mathcal{F}' := \Delta^* (\mathcal{F} \boxtimes \mathcal{F}')$.

The following diagram explains the word "restriction with supports":
% https://q.uiver.app/?q=WzAsNixbMiwwLCJLXkcoWiBcXGNhcCBaJykiXSxbMSwwLCJLXkcoWiBcXHRpbWVzIFonKSJdLFswLDAsIkteRyhaKSBcXHRpbWVzIEteRyhaJykiXSxbMCwxLCJLXkcoWCkgXFx0aW1lcyBLXkcoWCkiXSxbMSwxLCJLXkcoWCBcXHRpbWVzIFgpIl0sWzIsMSwiS15HKFgpIl0sWzEsMCwiXFxEZWx0YV4qIiwwLHsic3R5bGUiOnsiYm9keSI6eyJuYW1lIjoiZGFzaGVkIn19fV0sWzIsMSwiXFxib3h0aW1lcyJdLFszLDQsIlxcYm94dGltZXMiXSxbNCw1LCJcXERlbHRhXioiXSxbMiwzXSxbMSw0XSxbMCw1XV0=
\[\begin{tikzcd}
	{K^G(Z) \times K^G(Z')} & {K^G(Z \times Z')} & {K^G(Z \cap Z')} \\
	{K^G(X) \times K^G(X)} & {K^G(X \times X)} & {K^G(X)}
	\arrow["{\Delta^*}", dashed, from=1-2, to=1-3]
	\arrow["\boxtimes", from=1-1, to=1-2]
	\arrow["\boxtimes", from=2-1, to=2-2]
	\arrow["{\Delta^*}", from=2-2, to=2-3]
	\arrow[from=1-1, to=2-1]
	\arrow[from=1-2, to=2-2]
	\arrow[from=1-3, to=2-3]
\end{tikzcd}\]
\end{defn}
\begin{lemma}\label{lem:unit_of_tensor_product}
Let $X$ be a smooth variety, $Z \subseteq X$ be a closed $G$-subvariety, $\pi_Z:Z \longrightarrow \pt$ be the projection map. For any $\alpha \in K^G(Z)$, $\alpha \otimes \pi_Z^* 1_{\Rpt(G)} = \alpha$.
\end{lemma}

\begin{proof}
This comes from the definition of the tensor product.
\end{proof}

\subsection{Algebraic structures of $K$-theory} 
With enough tools in hand, we can define some extra structures on $K^G(X)$. (By priority $K^G(X)$ is an abelian group)

\begin{proposition}[$\Rpt(G)$-module]
For any $G$-variety $X$, $K^G(X)$ is a $\Rpt (G)$-module by
$$\Rpt (G) \times K^G(X) \cong K^G(\pt) \times K^G(X)  \begin{tikzcd}[column sep=5mm]
 	{} & {}
 	\arrow["{\boxtimes}", from=1-1, to=1-2]
 \end{tikzcd} 
 K^G(\pt \times X) \cong K^G(X).$$
\end{proposition}
Under this proposition, these three functors become $\Rpt(G)$-homomorphisms.

\begin{proposition}[$\otimes$ as multiplication]
For any smooth $G$-variety $X$, $K^G(X)$ is a unital commutative associative $\Rpt(G)$-algebra, where the multiplication (call the $\otimes$-product on $K^G(X)$) is defined by
$$K^G(X) \times K^G(X) 
\begin{tikzcd}[column sep=5mm]
 	{} & {}
 	\arrow["{\otimes}", from=1-1, to=1-2]
 \end{tikzcd}
 K^G(X).
 $$
\end{proposition}
Under this proposition, for any morphism $f:Y \longrightarrow X$ of smooth $G$-varieties, $f^*$ is a ring homomorphism.

\begin{warning}
We will define another product (called the convolution product) on some $K$-theories in Section \ref{sec:convolution}. These two products are essentially different products, and people have to specify which one they are using, when they discuss the "algebra structures on $K$-theories". The final task is to compute the convolution product of $K^{G_{\dimvec{d}}} (\St_{\dimvec{d}})$, not the $\otimes$-product.

After that, whenever we see an isomorphism of $K$-theories, we need to specify which structures this isomorphism preserve.
\end{warning}

\section{Thom isomorphism}
In this section we state Thom isomorphism theorem, which is an analogy of Poincaré lemma in $K$-theory.
\begin{proposition}[{Thom isomorphism, \cite[Theorem 5.4.17]{chriss1997representation}}]
Let $X$ be a $G$-variety, $\pi: E \longrightarrow X$ be a $G$-equivariant affine bundle on $X$. The pullback 
$$\pi^*: K^G(X) \longrightarrow K^G(E)$$
is an isomorphism of $K$-theories as $\Rpt(G)$-modules.
\end{proposition}
For a proof, see \cite[Theorem 5.4.17]{chriss1997representation}.

With Thom isomorphism, we can compute $K$-theory of affine bundles by the $K$-theory of the base spaces. Proposition \ref{prop:strataffine} offers plenty of cases to apply Thom isomorphism. Also, for any $k \in \mathbb{N}_{>0}$,
$$K^G(\mathbb{A}^k) \cong K^G(\pt)\cong \Rpt(G).$$
as an $\Rpt(G)$-module. This can be applied to $\OOmcell_w^u$ and $\OOmcell_{w,w'}^{u,u'}$. 




\section{Induction}
\subsection{Contracted product}\label{subsec:contracted_product}
Before we state the induction isomorphism, let us recall one basic construction of spaces: the contracted product.
\begin{defn}[Contracted product]
Let $H \subseteq G$ be a closed algebraic subgroup and $X$ be an $H$-variety. The contracted product of $G$ and $X$ over $H$ is defined as
$$G \times^H X :=(G \times X)/\sim$$
where
$$(gh,x) \sim (g,hx) \qquad \text{ for any $g \in G$, $h \in H$, $x \in X$.}$$
\end{defn}
$G \times^H X$ has a natural variety structure, which is not easy to construct. $G$ acts on $G \times^H X$ by multiplying from the left side. We have a $G$-equivariant flat morphism
$$G \times^H X \longrightarrow G/H \qquad (g,x) \longrightarrow gH$$
which realize $G \times^H X$ as an $X$-bundle over $G/H$. In particular, for $X=\pt$, we get an isomorphism of $G$-varieties
$$G \times^H \pt \stackrel{\sim}{\longrightarrow} G/H.$$

The contracted product is not only used for the induction isomorphism, but also used in the definition of equivariant cohomology theory (see Definition \ref{def:equivariant_cohomology}) and description of some typical varieties (see the description of $\overline{\OOmcell}_s$ in  \ref{subsec:product_of_F}).

\begin{eg}\label{eg:contracted_product_FF}
In the setting \ref{set:initial_case}, the $\GL_n$-equivariant map
$$\GL_n \times^B \mathcal{F} \stackrel{\sim}{\longrightarrow} \GL_n/B \times \mathcal{F} =\mathcal{F} \times \mathcal{F} \qquad (g,g'B) \longmapsto (gB,gg'B)$$
realizes $\mathcal{F} \times \mathcal{F}$ as a contracted product, and 
$$\OOmcell_{w'} \cong \GL_n \times^B \Omcell_{w'}$$
under this isomorphism.
\end{eg}
\subsection{Statement}

\begin{proposition}[{Induction isomorphism, \cite[5.2.16]{chriss1997representation}}]\label{prop:induction_isomorphism}
Let $H \subseteq G$ be a closed algebraic subgroup and $X$ be an $H$-variety, we have a Cartesian diagram of $H$-varieties
% https://q.uiver.app/?q=WzAsNCxbMCwwLCJYPUggXFx0aW1lc15IWCJdLFsxLDAsIkcgXFx0aW1lc15IWCJdLFswLDEsIlxccHQ9SC9IIl0sWzEsMSwiRy9IIl0sWzAsMl0sWzIsMywiXFxpb3RhX3tcXHB0fSJdLFsxLDMsIlxccGkiXSxbMCwxLCJcXGlvdGFfWCJdXQ==
\[\begin{tikzcd}
	{\makebox[10ex][r]{$X=H \times^HX$}} & {G \times^HX} \\
	{\makebox[5ex][r]{$\pt=H/H$}} & {G/H}
	\arrow[from=1-1, to=2-1]
	\arrow["{\iota_{\pt}}", from=2-1, to=2-2]
	\arrow["\pi", from=1-2, to=2-2]
	\arrow["{\iota_X}", from=1-1, to=1-2]
\end{tikzcd}\]

The functor 
% https://q.uiver.app/?q=WzAsMyxbMCwwLCJcXFJlc19IXkc6IFxcQ29oXkcoRyBcXHRpbWVzXntIfSBYKSJdLFsxLDAsIlxcQ29oXkgoRyBcXHRpbWVzXntIfSBYKSJdLFsyLDAsIlxcQ29oXkgoWCkiXSxbMSwyLCJcXGlvdGFfWF4qIl0sWzAsMSwiXFx0ZXh0e2ZvcmdldH0iXV0=
\[\begin{tikzcd}
	{\Res_H^G: \Coh^G(G \times^{H} X)} & {\Coh^H(G \times^{H} X)} & {\Coh^H(X)}
	\arrow["{\iota_X^*}", from=1-2, to=1-3]
	\arrow["{\text{forget}}", from=1-1, to=1-2]
\end{tikzcd}\]
is an equivalence of categories, and descend to an $\Rpt(H)$-module homomorphism of $K$-groups:
\[\begin{tikzcd}
	{\Res_H^G: K^G(G \times^{H} X)} & {K^H(G \times^{H} X)} & {K^H(X)}
	\arrow["{\iota_X^*}", from=1-2, to=1-3]
	\arrow["{\text{forget}}", from=1-1, to=1-2]
\end{tikzcd}\]
When $X$ is smooth, $\Res_H^G$ is an isomorphism as algebras (for $\otimes$-product).
\end{proposition}

We denote the inverse functor of $\Res_H^G$ by $\Ind_H^G$, called the induction, which is also explicitly constructed by pulling back and descent argument in \cite[5.2.16]{chriss1997representation}.

(???Present the construction of $\Ind_H^G$ and example of $K^{\GL_2} (\mathbb{P}^1)$, if time permits.)

\begin{remark}
The isomorphism $\Res_H^G$ also gives $K^G(G \times^H X)$ a $\Rpt(H)$-module structure.
\end{remark}
\subsection{Applications}
This induction formula is usually used for computing $G$-equivariant $K$-theory of $G$-orbits. For example, in Setting \ref{set:initial_case},
$$K^{\GL_n}(\mathcal{F})=K^{\GL_n}(\GL_n/B) \cong K^B(\pt)=\Rpt(B)$$
is an isomorphism as $\Rpt(\GL_n)$-modules. Notice that $K^{\GL_n}(\mathcal{F})$ is a free $\Rpt(\GL_n)$-module of rank $\#W=n!$.

Also, the isomorphism
$$K^{\GL_n}(\mathcal{F} \times \mathcal{F}) \cong K^{\GL_n}(\GL_n \times^B\mathcal{F}) \cong K^{B}(\mathcal{F})$$
gives $K^{\GL_n}(\mathcal{F} \times \mathcal{F})$ a $\Rpt(B)$-module structure.

In the next section we will explore how to reduce $B$-equivariant $K$-theory to $T$-equivariant $K$-theory.


 
\section{Reduction}
Let $P = M \ltimes U$ be a linear algebraic group in this section, where $M$ is reductive and $U=R_u(M)$ is the unipotent radical of $P$.

\begin{proposition}[{Reduction isomorphism, \cite[5.2.18]{chriss1997representation} }]\label{prop:reduction_isomorphism}
For any $P$-variety $X$, the forgetful map
$$K^P(X) \longrightarrow K^M(X)$$
is an isomorphism as $\Rpt(M)$-modules. (and as algebras for $\otimes$-product, when $X$ is smooth)
\end{proposition}

In the proof of reduction isomorphism, induction isomorphism and Thom isomorphism are used in an essential way.

This isomorphism allows us to identify $B$-equivariant $K$-theory and $T$-equivariant $K$-theory. In particular, $\Rpt(B) \cong \Rpt(T)$ as $\mathbb{Z}$-algebras.

\section{Equivariant cohomology theory}
The theory of equivariant cohomology theory is completely parallel with the theory of equivariant $K$-theory. We shortly sketch the definition and refer readers to see \cite[Chapter 2]{przezdziecki2015geometric} for details (like the definition of universal principle bundle $\EGG G \longrightarrow \BGG G$)

Nearly all the abstract results for $K$-theory have a corresponding cohomology theory version in \cite{przezdziecki2015geometric}. We will mention about the difference of Euler class in Section \ref{sec:euler_class}, compute some examples in Chapter \ref{chap:diagram}, and compare these two theories in Chapter \ref{chap:AScompletion}.

\subsection{$G$-equivariant cohomology $H_G^* (X; \mathbb{Q})$}

\begin{defn}[{$G$-equivariant cohomology, \cite[Definition 2.7]{przezdziecki2015geometric}}]\label{def:equivariant_cohomology}
For a $G$-variety $X$, the $G$-equivariant cohomology theory is defined as the cohomology ring of the contracted product space $\EGG G \times^G X$, i.e.,
$$H_G^* (X; \mathbb{Q}):= H^* \!\left(\EGG G \times^G X; \mathbb{Q}\right).$$
Specifically, for a point $\{\pt\}=\Spec \mathbb{C}$ with trivial $G$-action, denote
$$\Spt(G):= H_G^* \big(\{\pt\}; \mathbb{Q}\big)=H^* (\BGG G; \mathbb{Q})$$
as the cohomology ring of classifying space $\BGG G$.

We work with coefficient $\mathbb{Q}$ for simplicity, and we may omit $\mathbb{Q}$ for the convenience of writing and typing.
\end{defn}


Parallelly, there are two extreme situations worth mentioning about. 
When $G=\Id$, $\EGG G =\{\pt\}$. Therefore,
$$H_{\Id}^* (X; \mathbb{Q})= H^* \!\left(\{\pt\} \times^{\Id} X; \mathbb{Q}\right) \cong H^* (X; \mathbb{Q}).$$
When $G$ acts on $X$ trivially, we get
$$H_{G}^* (X; \mathbb{Q})= H^* (\BGG G \times X; \mathbb{Q}) \cong H^* (\BGG G; \mathbb{Q}) \otimes_{\mathbb{Q}} H^* (X; \mathbb{Q}).$$

\subsection{Cohomology ring $\Spt(G)$}\label{subsec:coh_ring}

We also list examples in parallel with subsection \ref{subset:rep_ring}. Everything is much more sketchy though. We use Setting \ref{set:initial_case}.

\begin{eg}
For trivial group $\Id$, $\BGG \Id=\{\pt\}$, so
$$\Spt(\Id)=H^*\big(\{\pt\} ; \mathbb{Q}\big) \cong \mathbb{Q}.$$
\end{eg}

\begin{eg}[{ \cite[Example 2.9(i)]{przezdziecki2015geometric} }]
For group $T$, $\BGG T=\prod_{j=i}^{n} \mathbb{CP}^{\infty}$, so
$$\Spt(T)=H^*\!\left(\prod_{j=1}^{n} \mathbb{CP}^{\infty} ; \mathbb{Q}\right) \cong \bigotimes_{j=1}^n H^*\left( \mathbb{CP}^{\infty}\; ; \mathbb{Q}\right) \cong \bigotimes_{j=1}^n \mathbb{Q}[t_j] = \mathbb{Q}[t_1,\ldots , t_n] $$
where $\deg t_j=2$ for any $j$.

By forgetting $T$-actions, we get a morphism of $\mathbb{Q}$-algebra
$$\Spt(T) \longrightarrow \Spt(\Id) \qquad f(t_1,\ldots,t_n) \longmapsto f(0,\ldots,0).$$
\end{eg}

\begin{eg}
By using the reduction isomorphism \ref{prop:reduction_isomorphism} in the version of cohomology theory, we can show that
$$\Spt(N) \cong \Spt(\Id) \cong \mathbb{Q} \qquad \Spt(B) \cong \Spt(T) \cong \mathbb{Q}\!\left[ t_1,\ldots,t_n \right]$$
\end{eg}

\begin{eg}[{\cite[Example 2.9(ii)]{przezdziecki2015geometric} }]
For group $\GL_n$, $\BGG \GL_n=\Gr(n,\infty)$, so
$$\Spt(T)=H^*\!\left(\Gr(n,\infty) ; \mathbb{Q}\right) \cong \mathbb{Q}[t_1,\ldots , t_n]^{S_n} $$

We also have the Chevalley restriction theorem in the version of cohomology theory. In this case, it says
$$\Spt(\GL_n) \cong \Spt(T)^{W} \cong \mathbb{Q}\!\left[ t_1,\ldots,t_n \right]^{S_n}.$$
\end{eg}

The three functors on cohomology theory are defined in a different way, which are induced from the three functors in normal cohomology theory, see \cite[2.3.2]{przezdziecki2015geometric}. Thom isomorphism, induction isomorphism and reduction isomorphism are still true in the equivariant cohomology theory case. In particular, we have
$$H_{\GL_n}^*(\mathcal{F}) \cong H_B^*(\pt) \cong H_T^*(\pt) \cong \mathbb{Q}\!\left[ t_1,\ldots,t_n \right]$$
as an $\Spt(\GL_n)$-module. $H_{\GL_n}^*(\mathcal{F})$ is a free $\Spt(\GL_n)$-module with rank $\#W=n!$.

Also, the isomorphism
$$H_{\GL_n}^*(\mathcal{F} \times \mathcal{F}) \cong H_{\GL_n}^*(\GL_n \times^B\mathcal{F}) \cong H_{B}^*(\mathcal{F})$$
gives $H_{\GL_n}^*(\mathcal{F} \times \mathcal{F})$ an $\Spt(B)$-module structure.
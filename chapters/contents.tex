%Name of the author of the thesis 
\authornew{Xiaoxiang Zhou}
%Date of birth of the Author
\geburtsdatum{9th March 1999}
%Place of Birth
\geburtsort{Ningde, China}
%Date of submission of the thesis
\date{4th January 2023}

%Name of the Advisor
% z.B.: Prof. Dr. Peter Koepke
\betreuer{Advisor: Prof. Dr. Catharina Stroppel}
%name of the second advisor of the thesis
\zweitgutachter{Second Advisor: Dr. Jens Niklas Eberhardt}

%Name of the Insitute of the advisor
%z.B.: Mathematisches Institut
\institut{Mathematical Institute}
%\institut{Institut f\"ur Angewandte Mathematik}
%\institut{Institut f\"ur Numerische Simulation}
%\institut{Forschungsinstitut f\"ur Diskrete Mathematik}
%Title of the thesis 
\title{Geometry of Quiver Flag Varieties}
%Do not change!
\ausarbeitungstyp{Master's Thesis  Mathematics}

\maketitle
\tableofcontents

%%%%%%%%%%%%%%%%%%%%%%%%%%%%%%%%%%%%%%%%%%%%%%%%%%%%%%%%%%%%%%%%%%%%%%%%%%%%%%%%%%%%%%%%%%%%%
\begin{warning}
I made some assumptions during the writing. To avoid confusing readers, these assumptions are listed here:
\begingroup
\setlist{itemsep=-0.4em}
\begin{itemize}
\item We use $\leqslant$ to represent subgroups and Bruhat orders. For example, $H \leqslant G$ means $H$ is a subgroup of $G$.

\item For the diagram, we always read from top to bottom.

\item For quivers, all the quivers we considered (except Auslander--Reiten quivers) are connected and finite (Remark \ref{rmk:quiver_restriction}). For simplicity, From Section \ref{sec:algebraic_group_Lie_algebra} to Chapter \ref{chap:excess_intersection_formula}, all the quivers have no loops or cycles.
%maybe the cycle condition can be removed.

\item  For any $\ww \in \WWd$, we always write $\ww = wu$, where $w \in \Wd$ and $u$ is the shortest element in the coset $\Wd \ww$. The flag-type dimension vector $\ftdimvec{d} \in \Wd \setminus \WWd$ corresponds to $u$, i.e., $\ftdimvec{d}=\Wd u$. Whenever $\tilde{w}$ and $\tilde{u}$ emerge, they are always defined by $uw'u'=\tilde{w}\tilde{u}$. See Section \ref{sec:algebraic_group_Lie_algebra}. 

\item Usually, the symbol $\ftdimvec{d}$ represents a complete   dimension vector in Part \ref{part:Steinberg_varieties}, while the symbols $\ftdimvec{f}, \ftdimvec{g}$ represent a (partial) dimension vector in Part \ref{part:partial_flag_varieties}.

\item We relable the coefficient ring before the basis $\preimage{\psi}_{\ww}$; see Remark \ref{rmk:relabling_of_coef}.
\end{itemize}
\endgroup
\end{warning}

\section*{Preprint and electronic version}
Over the course of this master project the second part of this thesis have appeared in the preprint \cite{zhou2022affine} on the arXiv. The updated version of this thesis can be found here:  \href{https://github.com/ramified/master_thesis/raw/main/master_thesis_Xiaoxiang_Zhou.pdf}{https://github.com/ramified/master\_thesis/raw/main/master\_thesis\_Xiaoxiang\_Zhou.pdf}
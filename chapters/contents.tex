

\maketitle
\tableofcontents

%%%%%%%%%%%%%%%%%%%%%%%%%%%%%%%%%%%%%%%%%%%%%%%%%%%%%%%%%%%%%%%%%%%%%%%%%%%%%%%%%%%%%%%%%%%%%
\begin{warning}
I made some assumptions during the writing. To avoid confusing readers, these assumptions are listed here:
\begin{itemize}
\item We use $\leqslant$ to represent subgroups and Bruhat orders. For example, $H \leqslant G$ means $H$ is a subgroup of $G$.

\item For the diagram, we always read from top to bottom.

\item For quivers, all the quivers we considered (except Auslander--Reiten quivers) are connected and finite (Remark \ref{rmk:quiver_restriction}). For simplicity, From Section \ref{sec:algebraic_group_Lie_algebra} to Chapter \ref{chap:excess_intersection_formula}, all the quivers have no loops or cycles.
%maybe the cycle condition can be removed.

\item  For any $\ww \in \WWd$, we always write $\ww = wu$, where $w \in \Wd$ and $u$ is the shortest element in the coset $\Wd \ww$. The flag-type dimension vector $\ftdimvec{d} \in \Wd \setminus \WWd$ corresponds to $u$, i.e., $\ftdimvec{d}=\Wd u$. Whenever $\tilde{w}$ and $\tilde{u}$ emerge, they are always defined by $uw'u'=\tilde{w}\tilde{u}$. See Section \ref{sec:algebraic_group_Lie_algebra}. 

\item We relable the coefficient ring before the basis $\preimage{\psi}_{\ww}$; see Remark \ref{rmk:relabling_of_coef}.
\end{itemize}
\end{warning}

\documentclass[reqno,11pt]{book}

%\usepackage{color,graphicx}
%\usepackage{mathrsfs,amsbsy}
\usepackage{amssymb}
\usepackage{amsmath}
\usepackage{amsfonts}
\usepackage{bm}
\usepackage{graphicx}
\usepackage{amsthm}
\usepackage{enumerate}
\usepackage[mathscr]{eucal}
\usepackage{float}
\usepackage{mathrsfs}
\usepackage{multicol}
\usepackage{multirow}
\usepackage[all,pdf]{xy}
\usepackage[a4paper,left=3cm,right=3cm]{geometry}
\usepackage[table,xcdraw]{xcolor} % before tikz-cd
\usepackage{tikz-cd}
\usepackage{hyperref}
\usepackage{scalerel}
\usepackage{stackengine,wasysym}
%\usepackage[notcite,notref]{showkeys}

% showkeys  make label explicit on the paper

\makeatletter
\@namedef{subjclassname@2010}{%
  \textup{2010} Mathematics Subject Classification}
\makeatother

\numberwithin{equation}{section}

\theoremstyle{plain}
\newtheorem{theorem}{Theorem}[section]
\newtheorem{lemma}[theorem]{Lemma}
\newtheorem{proposition}[theorem]{Proposition}
\newtheorem{corollary}[theorem]{Corollary}
\newtheorem{claim}[theorem]{Claim}
\newtheorem{defn}[theorem]{Definition}
\newtheorem{ques}[theorem]{Question}
\newtheorem*{bbox}{Black box}
\newtheorem{eg}[theorem]{Example}

\theoremstyle{plain}
\newtheorem{thmsub}{Theorem}[subsection]
\newtheorem{lemmasub}[thmsub]{Lemma}
\newtheorem{corollarysub}[thmsub]{Corollary}
\newtheorem{propositionsub}[thmsub]{Proposition}
\newtheorem{defnsub}[thmsub]{Definition}

\numberwithin{equation}{section}


\theoremstyle{remark}

\newtheorem{remark}[theorem]{Remark}
\newtheorem{remarks}{Remarks}

%\renewcommand\thefootnote{\fnsymbol{footnote}}
%dont use number as footnote symbol, use this command to change

\DeclareMathOperator{\GL}{\operatorname{GL}}
\DeclareMathOperator{\SL}{\operatorname{SL}}
\DeclareMathOperator{\supp}{supp}
\DeclareMathOperator{\dist}{dist}
\DeclareMathOperator{\vol}{vol}
\DeclareMathOperator{\diag}{diag}
\DeclareMathOperator{\tr}{tr}
\DeclareMathOperator{\Img}{\operatorname{Im}}
\DeclareMathOperator{\Id}{\operatorname{Id}}
\DeclareMathOperator{\Rep}{\operatorname{Rep}}
\DeclareMathOperator{\Mod}{\operatorname{Mod}}
\DeclareMathOperator{\Hom}{\operatorname{Hom}}
\DeclareMathOperator{\Ext}{\operatorname{Ext}}
\DeclareMathOperator{\gldim}{\operatorname{gl.dim}}
\DeclareMathOperator{\projdim}{\operatorname{proj.dim}}
\DeclareMathOperator{\injdim}{\operatorname{inj.dim}}
\DeclareMathOperator{\dimv}{\operatorname{\underline{\mathbf{dim}}}}
\DeclareMathOperator{\Pic}{\operatorname{Pic}}
\DeclareMathOperator{\Jac}{\operatorname{Jac}}
\newcommand{\Spec}{\operatorname{Spec}}
\DeclareMathOperator{\Flagd}{\operatorname{Flag}_{\mathbf{d}}}
\DeclareMathOperator{\Flagdstr}{\operatorname{Flag}_{\mathbf{d},str}}
\newcommand{\Gr}{\operatorname{Gr}}
\newcommand{\Grr}{\operatorname{Gr}}
\newcommand{\Grq}{\operatorname{Gr}^{KQ}}
\newcommand{\Flag}[1]{\operatorname{Flag}_{\mathbf{#1}}}
\newcommand{\Flagstr}[1]{\operatorname{Flag}_{\mathbf{#1},str}}
\newcommand{\dimvec}[1]{\mathbf{#1}}
\newcommand{\norm}[1]{\Vert{#1}\Vert}
\newcommand{\ord}{\operatorname{ord}}
\newcommand{\orde}{\operatorname{ord}_e }
\newcommand{\representation}[2]{\genfrac{}{}{0pt}{3}{\phantom{000}#2\phantom{00}}{#1}}
\newcommand{\Vcell}{\operatorname{\mathcal{V}}}
\setlength\intextsep{0cm}
\setlength\textfloatsep{0cm}


% from https://tex.stackexchange.com/questions/63545/big-tilde-in-math-mode

\begin{document}
\date{}

\title
{Master thesis
}


%\author{Xiaoxiang Zhou}
%\address{School of Mathematical Sciences\\
%University of Bonn\\
%Bonn, 53115\\ Germany\\} 
%\email{email:xx352229@mail.ustc.edu.cn}



\setcounter{tocdepth}{1}
\maketitle
\tableofcontents
%%%%%%%%%%%%%%%%%%%%%%%%%%%%%%%%%%%%%%%%%%%%%%%%%%%%%%%%%%%%%%%%%%%%%%%%%%%%%%%%%%%%%%%%%%%%%
\chapter{Variety and stratification}

\section{Initial case: $\mathcal{F}$ and $\mathcal{F} \times \mathcal{F}$}

We introduce the complete flag variety to give a bird's eye view on the whole section. Actually, the entire difficulty is bundled in this example.

Fix $n \geqslant 1$, we denote $\GL_n:=\GL_n(\mathbb{C})$, $B$, $T$, $N$, $W$ be the standard Borel subgroup, standard torus, unipotent subgroup, Weyl group respectively, i.e.,

$$\GL_n= \begin{pmatrix}
* & \dots & * \\
\vdots & \ddots & \vdots \\
* & \dots & * 
\end{pmatrix} \quad
B= \begin{pmatrix}
* & \dots & * \\
 & \ddots & \vdots \\
0 &  & * 
\end{pmatrix} \quad
T= \begin{pmatrix}
* &  & 0 \\
 & \ddots &  \\
0 &  & * 
\end{pmatrix} \quad
B= \begin{pmatrix}
1 & \dots & * \\
 & \ddots & \vdots \\
0 &  & 1
\end{pmatrix} \quad
$$
$$W:= N_{\GL_n}(T)/T \cong S_n$$

\begin{defn}[flag]
For a finite dimensional $\mathbb{C}$-vector space $V$, a flag of $V$ is an increasing sequence of subspaces of $V$:
$$F: 0 \subseteq V_1 \subseteq V_2 \subseteq \cdots \subseteq V_k = \mathbb{C}^n.$$
$F$ is called a complete flag if $\dim V_i = i$ for all $i$, otherwise $F$ is called a partial flag.
\end{defn}

\begin{defn}[complete flag variety]
The complete flag variety $\mathcal{F}$ is defined as 
\begin{equation*}
\begin{aligned}
  \mathcal{F}=\;& \GL_n/B \\ 
    \cong\;&  \left\{ \text{complete flags of } \mathbb{C}^n \right\} \\ 
      =\;& \left\{ 0 \subseteq V_1 \subseteq V_2 \subseteq \cdots \subseteq V_n = \mathbb{C}^n \;\middle|\; \dim V_i =i \right\} \\ 
       \cong\;&  \left\{ \text{Borel subgroups of } \GL_n \right\} \\
          =\;& \left\{ gBg^{-1} \;\middle|\; g \in \GL_n \right\} \\ 
\end{aligned}
\end{equation*}
\end{defn}

\begin{remark}\
\begin{enumerate}
\item We implicitly give the base point of $\mathcal{F}$, which is not considered as the data of $\mathcal{F}$. Fix a standard basis of $\mathbb{C}^n$ by $\{v_1, \ldots, v_n \}$, we define the standard flag
$$F_{\Id}: 0 \subseteq \left< v_1 \right> \subseteq \left< v_1,v_2 \right> \subseteq \cdots \subseteq \left< v_1, \ldots, v_n \right> = \mathbb{C}^n.$$
\item We have the natural $\GL_n$-action on $\mathcal{F}$, which is considered as the data of $\mathcal{F}$.

For $g \in \GL_n$, we define the flag attached to $g$: 
$$F_g \triangleq gF_{\Id}: 0 \subseteq \left< gv_1 \right> \subseteq \left< gv_1,gv_2 \right> \subseteq \cdots \subseteq \left< gv_1, \ldots, gv_n \right> = \mathbb{C}^n.$$
Especially, for $w \in W=N_{\GL_n}(T)/T \cong S_n$, the flag attached to $w$
\begin{equation*}
\begin{aligned}
  F_w:\;&  0 \subseteq \left< \tilde{w}v_1 \right> \subseteq \left< \tilde{w}v_1,\tilde{w}v_2 \right> \subseteq \cdots \subseteq \left< \tilde{w}v_1, \ldots, \tilde{w}v_n \right> = \mathbb{C}^n\\ 
  &  0 \subseteq \left< v_{w(1)} \right> \subseteq \left< v_{w(1)},v_{w(2)} \right> \subseteq \cdots \subseteq \left< v_{w(1)}, \ldots, v_{w(n)} \right> = \mathbb{C}^n\\
\end{aligned}
\end{equation*}
does not depend on the choice of the lift $\tilde{w} \in N_{\GL_n}(T)$ of $w$.

Readers can verify that $\left\{ F_w \middle| w \in W \right\}$ are all $T$-fixed points of $\mathcal{F}$, while $\left\{ wBw^{-1} \middle| w \in W \right\}$ are all Borel subgroups of $G$ containing the standard torus $T$.
\item For $n=2$, $\mathcal{F} \cong \mathbb{P}^1$. We encourage readers to use $\mathbb{P}^1$ as a toy example for the whole theory.
\end{enumerate}
\end{remark}
\begin{table}[ht]
\centering
\begin{tabular}{|c|c|c|c|}
\hline
interpretation & $\GL_n/B$           & flags                       & Borel subgroups \\ \hline
base point     & $\Id$               & $F_{\Id}$                   & $B$             \\ \hline
$\GL_n$-action & left multiplication & $\{V_i\} \mapsto \{gV_i \}$ & conjugation     \\ \hline
general point  & $g$                 & $F_g$                       & $gBg^{-1}$      \\ \hline
\end{tabular}
\end{table}
        
$\mathcal{F}$ is a well-studied variety, and has many combinatorical properties. For example, from the well-known Bruhat decomposition, \footnote{For the most time the formula doees not depend on the lift of $w$, so we abuse the notation of $w \in N_{\GL_n}(T)/T$ and $\tilde{w} \in N_{\GL_n}(T)$.}
$$\GL_n \cong \bigsqcup_{w\in W} BwB$$

We get a stratification of $\mathcal{F}$ by $B$-orbits:
$$\mathcal{F} = \GL_n/B \cong \bigsqcup_{w\in W} BwB/B$$
The $B$-orbit $BwB/B$ is called the Schubert cell, denoted by $\Vcell_w$. Since 
$$\Vcell_w=BwB/B \cong B/\left( B \cap wBw^{-1} \right) \cong \mathbb{A}^{l(w)},$$
the Schubert cell is an affine space of dimension $l(w)$.

As a result, we know a lot of information of $\mathcal{F}$:
%https://tex.stackexchange.com/questions/112576/math-mode-in-tabular-without-having-to-use-everywhere
\newcolumntype{C}{>{$}c<{$}} % math-mode version of "l" column type
\begin{table}[]
\centering
\begin{tabular}{|C|C|C|C|C|C|C|C|}
\hline
H^i(\mathcal{F}; \mathbb{C}) & 0 & 2 & 4 & 6  & 8  & 10 & 12 \\ \hline
1                            & 1 &   &   &    &    &    &    \\ \hline
2                            & 1 & 1 &   &    &    &    &    \\ \hline
3                            & 1 & 2 & 2 & 1  &    &    &    \\ \hline
4                            & 1 & 3 & 5 & 6  & 5  & 3  & 1  \\ \hline
5                            & 1 & 4 & 9 & 15 & 20 & 22 & 20 \\ \hline
\end{tabular}
\end{table}
\begin{table}[]
\centering
\begin{tabular}{C|C|C}
\hline
G    & \text{Orbit}                       & G\text{-fixed points}  \\ \hline
GL_n & \mathcal{F} \cong \GL_n/B          & \varnothing            \\ \hline
B    & \Vcell_w \cong B/(B \cap wBw^{-1}) & \{F_{\Id} \}           \\ \hline
T    & -                                  & \{F_w |  \, w \in W \} \\ \hline
\end{tabular}
\end{table}
\section{quiver and Weyl group}
\section{algebraic group and Lie algebra}
\section{typical variety}
\section{stratification and $T$-fixed points}
%%%%%%%%%%%%%%%%%%%%%%%%%%%%%%%%%%%%%%%%%%%%%%%%%%%%%%%%%%%%%%%%%%%%%%%%%%%%%%%%%%%%%%%%%%%%%
\chapter{K-theory and cohomology theory}
%%%%%%%%%%%%%%%%%%%%%%%%%%%%%%%%%%%%%%%%%%%%%%%%%%%%%%%%%%%%%%%%%%%%%%%%%%%%%%%%%%%%%%%%%%%%%
\chapter{Localization theorem}

%%%%%%%%%%%%%%%%%%%%%%%%%%%%%%%%%%%%%%%%%%%%%%%%%%%%%%%%%%%%%%%%%%%%%%%%%%%%%%%%%%%%%%%%%%%%%
\chapter{Excess intersection formula}

%%%%%%%%%%%%%%%%%%%%%%%%%%%%%%%%%%%%%%%%%%%%%%%%%%%%%%%%%%%%%%%%%%%%%%%%%%%%%%%%%%%%%%%%%%%%%
\chapter{From formula to diagram}


%%%%%%%%%%%%%%%%%%%%%%%%%%%%%%%%%%%%%%%%%%%%%%%%%%%%%%%%%%%%%%%%%%%%%%%%%%%%%%%%%%%%%%%%%%%%%
\chapter{Generalization}

%%%%%%%%%%%%%%%%%%%%%%%%%%%%%%%%%%%%%%%%%%%%%%%%%%%%%%%%%%%%%%%%%%%%%%%%%%%%%%%%%%%%%%%%%%%%%
\chapter{Atiyah-Segal completion theorem}


%\bibliographystyle{plain}
%\bibliography{reference}
\end{document}





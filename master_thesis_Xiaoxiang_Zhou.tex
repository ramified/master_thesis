
\documentclass[reqno,11pt]{book}

%\usepackage{color,graphicx}
%\usepackage{mathrsfs,amsbsy}
\usepackage{amssymb}
\usepackage{amsmath}
\usepackage{amsfonts}
\usepackage{bm}
\usepackage{graphicx}
\usepackage{amsthm}
\usepackage{enumerate}
\usepackage{extarrows}
\usepackage[mathscr]{eucal}
\usepackage{float}
\usepackage{mathrsfs}
\usepackage{multicol}
\usepackage{multirow}
\usepackage[all,pdf]{xy}
\usepackage[a4paper,left=3cm,right=3cm]{geometry}
\usepackage[table,xcdraw]{xcolor} % before tikz-cd
\usepackage{tikz-cd}
\usepackage{hyperref}
\usepackage{scalerel}
\usepackage{stackengine,wasysym}
%\usepackage[notcite,notref]{showkeys}

% showkeys  make label explicit on the paper

\makeatletter
\@namedef{subjclassname@2010}{%
  \textup{2010} Mathematics Subject Classification}
\makeatother

\numberwithin{equation}{section}

\theoremstyle{plain}
\newtheorem{theorem}{Theorem}[section]
\newtheorem{lemma}[theorem]{Lemma}
\newtheorem{proposition}[theorem]{Proposition}
\newtheorem{corollary}[theorem]{Corollary}
\newtheorem{claim}[theorem]{Claim}
\newtheorem{defn}[theorem]{Definition}
\newtheorem{ques}[theorem]{Question}
\newtheorem*{bbox}{Black box}
\newtheorem{eg}[theorem]{Example}

\theoremstyle{plain}
\newtheorem{thmsub}{Theorem}[subsection]
\newtheorem{lemmasub}[thmsub]{Lemma}
\newtheorem{corollarysub}[thmsub]{Corollary}
\newtheorem{propositionsub}[thmsub]{Proposition}
\newtheorem{defnsub}[thmsub]{Definition}

\numberwithin{equation}{section}


\theoremstyle{remark}

\newtheorem{remark}[theorem]{Remark}
\newtheorem{remarks}{Remarks}

%\renewcommand\thefootnote{\fnsymbol{footnote}}
%dont use number as footnote symbol, use this command to change

\DeclareMathOperator{\GL}{\operatorname{GL}}
\DeclareMathOperator{\SL}{\operatorname{SL}}
\DeclareMathOperator{\supp}{supp}
\DeclareMathOperator{\dist}{dist}
\DeclareMathOperator{\vol}{vol}
\DeclareMathOperator{\diag}{diag}
\DeclareMathOperator{\tr}{tr}
\DeclareMathOperator{\Img}{\operatorname{Im}}
\DeclareMathOperator{\Id}{\operatorname{Id}}
\DeclareMathOperator{\Rep}{\operatorname{Rep}}
\DeclareMathOperator{\Mod}{\operatorname{Mod}}
\DeclareMathOperator{\Hom}{\operatorname{Hom}}
\DeclareMathOperator{\Ext}{\operatorname{Ext}}
\DeclareMathOperator{\gldim}{\operatorname{gl.dim}}
\DeclareMathOperator{\projdim}{\operatorname{proj.dim}}
\DeclareMathOperator{\injdim}{\operatorname{inj.dim}}
\DeclareMathOperator{\dimv}{\operatorname{\underline{\mathbf{dim}}}}
\DeclareMathOperator{\Pic}{\operatorname{Pic}}
\DeclareMathOperator{\Jac}{\operatorname{Jac}}
\newcommand{\Spec}{\operatorname{Spec}}
\DeclareMathOperator{\Flagd}{\operatorname{Flag}_{\mathbf{d}}}
\DeclareMathOperator{\Flagdstr}{\operatorname{Flag}_{\mathbf{d},str}}
\newcommand{\Gr}{\operatorname{Gr}}
\newcommand{\Grr}{\operatorname{Gr}}
\newcommand{\Grq}{\operatorname{Gr}^{KQ}}
\newcommand{\Flag}[1]{\operatorname{Flag}_{\mathbf{#1}}}
\newcommand{\Flagstr}[1]{\operatorname{Flag}_{\mathbf{#1},str}}
\newcommand{\dimvec}[1]{\mathbf{#1}}
\newcommand{\abdimvec}[1]{|\dimvec{#1}|}
\newcommand{\absgp}[1]{\mathbb{#1}}
\newcommand{\ww}{\varpi}
\newcommand{\norm}[1]{\Vert{#1}\Vert}
\newcommand{\ord}{\operatorname{ord}}
\newcommand{\orde}{\operatorname{ord}_e }
\newcommand{\representation}[2]{\genfrac{}{}{0pt}{3}{\phantom{000}#2\phantom{00}}{#1}}
\newcommand{\Vcell}{\operatorname{\mathcal{V}}}
\setlength\intextsep{0cm}
\setlength\textfloatsep{0cm}


% from https://tex.stackexchange.com/questions/63545/big-tilde-in-math-mode

\begin{document}
\date{}

\title
{Master thesis
}


%\author{Xiaoxiang Zhou}
%\address{School of Mathematical Sciences\\
%University of Bonn\\
%Bonn, 53115\\ Germany\\} 
%\email{email:xx352229@mail.ustc.edu.cn}



\setcounter{tocdepth}{1}
\maketitle
\tableofcontents
%%%%%%%%%%%%%%%%%%%%%%%%%%%%%%%%%%%%%%%%%%%%%%%%%%%%%%%%%%%%%%%%%%%%%%%%%%%%%%%%%%%%%%%%%%%%%
\chapter{Variety and stratification}

\section{Initial case: $\mathcal{F}$ and $\mathcal{F} \times \mathcal{F}$}

We introduce the complete flag variety to give a bird's eye view on the whole section. Actually, the entire difficulty is bundled in this example.

Fix $n \geqslant 1$, we denote $\GL_n:=\GL_n(\mathbb{C})$, $B$, $T$, $N$, $W$ be the standard Borel subgroup, standard torus, unipotent subgroup, Weyl group respectively, i.e.,
{
\setlength\arraycolsep{1pt}
\renewcommand{\arraystretch}{0.6}
$$\GL_n= \begin{pmatrix}
* & \cdots & * \\[-1.4mm]
\vdots & \ddots & \vdots \\
* & \cdots & * 
\end{pmatrix} \quad
B= \begin{pmatrix}
* & \cdots & * \\[-1.4mm]
 & \ddots & \vdots \\
\scriptstyle 0 &  & * 
\end{pmatrix} \quad
T= \begin{pmatrix}
* &  & \scriptstyle 0 \\[-1.4mm]
 & \ddots &  \\
\scriptstyle 0 &  & * 
\end{pmatrix} \quad
B= \begin{pmatrix}
\scriptstyle 1 & \cdots & * \\[-1.4mm]
 & \ddots & \vdots \\
\scriptstyle 0 &  & \scriptstyle 1
\end{pmatrix} \quad
$$
$$W:= N_{\GL_n}(T)/T \cong S_n$$
}

\begin{defn}[flag]
For a finite dimensional $\mathbb{C}$-vector space $V$, a flag of $V$ is an increasing sequence of subspaces of $V$:
$$F: 0 \subseteq V_1 \subseteq V_2 \subseteq \cdots \subseteq V_k = \mathbb{C}^n.$$
$F$ is called a complete flag if $\dim V_i = i$ for all $i$, otherwise $F$ is called a partial flag.
\end{defn}

\begin{defn}[complete flag variety]
The complete flag variety $\mathcal{F}$ is defined as 
\begin{equation*}
\begin{aligned}
  \mathcal{F}=\;& \GL_n/B \\ 
    \cong\;&  \left\{ \text{complete flags of } \mathbb{C}^n \right\} \\ 
      =\;& \left\{ 0 \subseteq V_1 \subseteq V_2 \subseteq \cdots \subseteq V_n = \mathbb{C}^n \;\middle|\; \dim V_i =i \right\} \\ 
       \cong\;&  \left\{ \text{Borel subgroups of } \GL_n \right\} \\
          =\;& \left\{ gBg^{-1} \;\middle|\; g \in \GL_n \right\} \\ 
\end{aligned}
\end{equation*}
\end{defn}

\begin{remark}\
\begin{enumerate}
\item We implicitly give the base point of $\mathcal{F}$, which is not considered as the data of $\mathcal{F}$. Fix a standard basis of $\mathbb{C}^n$ by $\{v_1, \ldots, v_n \}$, we define the standard flag
$$F_{\Id}: 0 \subseteq \left< v_1 \right> \subseteq \left< v_1,v_2 \right> \subseteq \cdots \subseteq \left< v_1, \ldots, v_n \right> = \mathbb{C}^n.$$
\item We have the natural $\GL_n$-action on $\mathcal{F}$, which is considered as the data of $\mathcal{F}$.

For $g \in \GL_n$, we define the flag attached to $g$: 
$$F_g \triangleq gF_{\Id}: 0 \subseteq \left< gv_1 \right> \subseteq \left< gv_1,gv_2 \right> \subseteq \cdots \subseteq \left< gv_1, \ldots, gv_n \right> = \mathbb{C}^n.$$
Especially, for $w \in W=N_{\GL_n}(T)/T \cong S_n$, the flag attached to $w$
\begin{equation*}
\begin{aligned}
  F_w:\;&  0 \subseteq \left< \tilde{w}v_1 \right> \subseteq \left< \tilde{w}v_1,\tilde{w}v_2 \right> \subseteq \cdots \subseteq \left< \tilde{w}v_1, \ldots, \tilde{w}v_n \right> = \mathbb{C}^n\\ 
  &  0 \subseteq \left< v_{w(1)} \right> \subseteq \left< v_{w(1)},v_{w(2)} \right> \subseteq \cdots \subseteq \left< v_{w(1)}, \ldots, v_{w(n)} \right> = \mathbb{C}^n\\
\end{aligned}
\end{equation*}
does not depend on the choice of the lift $\tilde{w} \in N_{\GL_n}(T)$ of $w$.

Readers can verify that $\left\{ F_w \middle| w \in W \right\}$ are all $T$-fixed points of $\mathcal{F}$, while $\left\{ wBw^{-1} \middle| w \in W \right\}$ are all Borel subgroups of $G$ containing the standard torus $T$.
\item For $n=2$, $\mathcal{F} \cong \mathbb{P}^1$. We encourage readers to use $\mathbb{P}^1$ as a toy example for the whole theory.
\end{enumerate}
\end{remark}
\begin{table}[ht]
\centering
\begin{tabular}{|c|c|c|c|}
\hline
interpretation & $\GL_n/B$           & flags                       & Borel subgroups \\ \hline
base point     & $\Id$               & $F_{\Id}$                   & $B$             \\ \hline
$\GL_n$-action & left multiplication & $\{V_i\} \mapsto \{gV_i \}$ & conjugation     \\ \hline
general point  & $g$                 & $F_g$                       & $gBg^{-1}$      \\ \hline
\end{tabular}
\end{table}
        
$\mathcal{F}$ is a well-studied variety, and has many combinatorical properties. For example, from the well-known Bruhat decomposition, \footnote{For the most time the formula doees not depend on the lift of $w$, so we abuse the notation of $w \in N_{\GL_n}(T)/T$ and $\tilde{w} \in N_{\GL_n}(T)$.}
$$\GL_n \cong \bigsqcup_{w\in W} BwB$$

We get a stratification of $\mathcal{F}$ by $B$-orbits:
$$\mathcal{F} = \GL_n/B \cong \bigsqcup_{w\in W} BwB/B$$
The $B$-orbit $BwB/B$ is called the Schubert cell, denoted by $\Vcell_w$. Since 
$$\Vcell_w=BwB/B \cong B/\left( B \cap wBw^{-1} \right) \cong \mathbb{A}^{l(w)},$$
the Schubert cell is an affine space of dimension $l(w)$.

As a result, we know a lot of information of $\mathcal{F}$:
%https://tex.stackexchange.com/questions/112576/math-mode-in-tabular-without-having-to-use-everywhere
\newcolumntype{C}{>{$}c<{$}} % math-mode version of "l" column type
\begin{table}[]
\centering
\begin{tabular}{|C|C|C|C|C|C|C|C|}
\hline
H^i(\mathcal{F}; \mathbb{C}) & 0 & 2 & 4 & 6  & 8  & 10 & 12 \\ \hline
1                            & 1 &   &   &    &    &    &    \\ \hline
2                            & 1 & 1 &   &    &    &    &    \\ \hline
3                            & 1 & 2 & 2 & 1  &    &    &    \\ \hline
4                            & 1 & 3 & 5 & 6  & 5  & 3  & 1  \\ \hline
5                            & 1 & 4 & 9 & 15 & 20 & 22 & 20 \\ \hline
\end{tabular}
\end{table}
\begin{table}[]
\centering
\begin{tabular}{C|C|C}
\hline
G    & \text{Orbit}                       & G\text{-fixed points}  \\ \hline
GL_n & \mathcal{F} \cong \GL_n/B          & \varnothing            \\ \hline
B    & \Vcell_w \cong B/(B \cap wBw^{-1}) & \{F_{\Id} \}           \\ \hline
T    & -                                  & \{F_w |  \, w \in W \} \\ \hline
\end{tabular}
\end{table}
\section{quiver and Weyl group}
\section{algebraic group and Lie algebra}

In this subsection we fix notations of algebraic group and Lie algebras. Later, the algebraic group will act on varieties, and some Lie algebra will serve as tangent spaces.

We fix a quiver $Q$, a dimension vector $\dimvec{d}$ and a $\mathbb{C}$-vector space with quiver partition
$$V=\bigoplus_{i \in Q_0} V_i \qquad \text{with } V_i=\mathbb{C}^{\dimvec{d}_i}.$$

\begin{defn}[absolute algebraic groups]
%The following algebraic groups are not related to the quiver partition of $V$.
We set
$$\absgp{G}_{\abdimvec{d}}:= \GL(V)=\GL_{\abdimvec{d}}(\mathbb{C}),$$
and $\absgp{B}_{\abdimvec{d}}$, $\absgp{T}_{\abdimvec{d}}$, $\absgp{N}_{\abdimvec{d}}$ are corresponding standard Borel, torus and unipotent subgroups.

The Weyl group is
$$\absgp{W}_{\abdimvec{d}}:= N_{\absgp{G}_{\abdimvec{d}}}(\absgp{T}_{\abdimvec{d}})/\absgp{T}_{\abdimvec{d}} \cong S_{\abdimvec{d}}.$$

For $\ww \in \absgp{W}_{\abdimvec{d}}$, we define\footnote{As usual, we abuse the notation of $\ww$ and its lift.}
$$\absgp{B}_{\ww}:= \ww \absgp{B}_{\abdimvec{d}} \ww^{-1}.$$
We will view $\absgp{B}_{\ww}$ as the stabilizer of the flag $F_{\ww}$ with $\absgp{G}_{\abdimvec{d}}$-action.
\end{defn}

We also have a series of algebraic groups compatible with the quiver partition of $V$, and they're more common in this thesis.

\begin{defn}[relative algebraic groups]

We set
$$G_{\dimvec{d}}:= \bigoplus_{i \in Q_0}\GL(V_i)=\GL_{\dimvec{d}_i}(\mathbb{C}) \subseteq \absgp{G}_{\abdimvec{d}},$$
and $B_{\dimvec{d}}$, $T_{\dimvec{d}}$, $N_{\dimvec{d}}$ are corresponding standard Borel, torus and unipotent subgroups.

The Weyl group is
$$W_{\dimvec{d}}:= N_{G_{\dimvec{d}}}(T_{\dimvec{d}})/T_{\dimvec{d}} \cong \prod_{i \in Q_0}S_{\dimvec{d}_i}.$$

For $\ww=wu \in W_{\dimvec{d}}$, we define
$$B_{\ww}:= w B_{\dimvec{d}} w^{-1}.$$
We will view $B_{\ww}$ as the stabilizer of the flag $F_{\ww}$ with $G_{\dimvec{d}}$-action.
\end{defn}

We also have a series of algebraic groups with subscription as elements in the Weyl group:
\begin{defn}[more algebraic groups]
For $\ww, \ww'' \in \absgp{W}_{\abdimvec{d}}$, define
\begin{equation*}
\begin{aligned}
  N_{\ww}:=\;& R_u(B_{\ww}),  \\ 
  N_{\ww,\ww''}:=\;& N_{\ww} \cap N_{\ww''},  \\ 
  M_{\ww,\ww''}:=\;& N_{\ww}/N_{\ww,\ww''},  \\ 
\end{aligned}
\end{equation*}
where $R_u$ denotes for the unipotent radical.

For $s \in \Pi$ such that $\ww s \ww^{-1} \in W_d$ (i.e., $W_{\dimvec{d}}\ww=W_{\dimvec{d}}\ww s$), define
\begin{equation*}
\begin{aligned}
  P_{\ww,\ww s}:\xlongequal{\ww =wu}\;& w\left( B_{\dimvec{d}} usu^{-1} B_{\dimvec{d}} \cup B_{\dimvec{d}} \right)w^{-1} \\ 
  \xlongequal{\phantom{\ww =wu}}\;&  B_{\ww} \ww s\ww^{-1} B_{\ww} \cup B_{\ww}  \\ 
\end{aligned}
\end{equation*}
\end{defn}
\begin{remark}
One can easily show that $N_{\ww, \ww s}=R_u (P_{\ww, \ww s})$.
\end{remark}
%problems on making short vertical space
%https://tex.stackexchange.com/questions/183562/how-to-reduce-vertical-space-in-matrix?newreg=ca78ba434426423a8be8ce4f1df4548c
%https://tex.stackexchange.com/questions/643384/horizontal-spacing-bmatrix-in-align-environment
%https://tex.stackexchange.com/questions/275725/adjusting-separation-between-matrix-entries
\setlength\arraycolsep{1pt}
\renewcommand{\arraystretch}{0.6}

%small fonts: use \scriptstyle instead of \scriptsize

%vertical line:
%https://tex.stackexchange.com/questions/33519/vertical-line-in-matrix-using-latexit

\makeatletter
\renewcommand*\env@matrix[1][*\c@MaxMatrixCols c]{%
  \hskip -\arraycolsep
  \let\@ifnextchar\new@ifnextchar
  \array{#1}}
\makeatother
\begin{eg}
For $\abdimvec{d}=5$, $\dimvec{d}=(3,2)$, ???,
\begin{equation*}
\begin{aligned}
  &\absgp{G}_{\abdimvec{d}}= \begin{pmatrix}
  * & * & * & * & * \\
  * & * & * & * & * \\
  * & * & * & * & * \\
  * & * & * & * & * \\
  * & * & * & * & * \\
  \end{pmatrix}&
  \absgp{B}_{\abdimvec{d}}=
  \begin{pmatrix}
  * & * & * & * & * \\
   & * & * & * & * \\
   &  & * & * & * \\
   &  &  & * & * \\
   &  &  &  & * \\
  \end{pmatrix}&&
  \absgp{T}_{\abdimvec{d}}=
  \begin{pmatrix}
  * &  &  &  &  \\
   & * &  &  &  \\
   &  & * &  &  \\
   &  &  & * &  \\
   &  &  &  & * \\
  \end{pmatrix}&&
  \absgp{N}_{\abdimvec{d}}=
  \begin{pmatrix}
  \scriptstyle 1 & * & * & * & * \\
   & \scriptstyle 1 & * & * & * \\
   &  & \scriptstyle 1 & * & * \\
   &  &  & \scriptstyle 1 & * \\
   &  &  &  & \scriptstyle 1 \\
  \end{pmatrix}
    \\
  &\absgp{W}_{\abdimvec{d}}\cong S_5&
  \absgp{B}_{\ww}=
   \begin{pmatrix}
  * & * &  &  & * \\
   & * &  &  &  \\
  * & * & * &  & * \\
  * & * & * & * & * \\
   & * &  &  & * \\
  \end{pmatrix}&&
  \absgp{B}_{\ww s}=
     \begin{pmatrix}
    * & * & *  &  & * \\
     & * &  &  &  \\
     & * & * &  & * \\
    * & * & * & * & * \\
     & * &  &  & * \\
    \end{pmatrix}&&
  \\
  &G_{\dimvec{d}}\;=
  \begin{pmatrix}[ccc|cc]
  * & * & * &  &  \\
  * & * & * &  &  \\
  * & * & * &  &  \\
     \hline
   &  &  & * & * \\ 
   &  &  & * & * \\ 
  \end{pmatrix}&
  B_{\dimvec{d}}=
    \begin{pmatrix}[ccc|cc]
    * & * & * &  &  \\
     & * & * &  &  \\
     &  & * &  &  \\
        \hline
     &  &  & * & * \\ 
     &  &  &  & * \\ 
    \end{pmatrix}&&
  T_{\dimvec{d}}=
    \begin{pmatrix}[ccc|cc]
     * &  &  &  &  \\
       & * &  &  &  \\
       &  & * &  &  \\
          \hline
       &  &  & * &  \\
       &  &  &  & * \\
    \end{pmatrix}&&
  N_{\dimvec{d}}=
         \begin{pmatrix}[ccc|cc]
          \scriptstyle 1 & * & * &  &  \\
           & \scriptstyle 1 & * &  &  \\
           &  & \scriptstyle 1 &  &  \\
              \hline
           &  &  & \scriptstyle 1 & * \\ 
           &  &  &  & \scriptstyle 1 \\ 
          \end{pmatrix}  
  \\
&W_{\dimvec{d}}\cong S_3 \times S_2&
  B_{\ww}=
   \begin{pmatrix}[ccc|cc]
  * & * &  &  &  \\
   & * &  &  &  \\
  * & * & * &  &  \\
     \hline
   &  &  & * &  \\
   &  &  & * & * \\
  \end{pmatrix}&&
  B_{\ww s}=
     \begin{pmatrix}[ccc|cc]
  * &  &  &  &  \\
  * & * &  &  &  \\
  * & * & * &  &  \\
     \hline
   &  &  & * &  \\
   &  &  & * & * \\
    \end{pmatrix}&&
  \\
   &N_{\ww}=
    \begin{pmatrix}[ccc|cc]
   \scriptstyle 1 & * &  &  &  \\
    & \scriptstyle 1 &  &  &  \\
   * & * & \scriptstyle 1 &  &  \\
      \hline
    &  &  & \scriptstyle 1 &  \\
    &  &  & * & \scriptstyle 1 \\
   \end{pmatrix}&
   N_{\ww,\ww s}=
      \begin{pmatrix}[ccc|cc]
   \scriptstyle 1 &  &  &  &  \\
    & \scriptstyle 1 &  &  &  \\
   * & * & \scriptstyle 1 &  &  \\
      \hline
    &  &  & \scriptstyle 1 &  \\
    &  &  & * & \scriptstyle 1 \\
     \end{pmatrix}&& 
   M_{\ww,\ww s}=
      \begin{pmatrix}[ccc|cc]
   \scriptstyle 1 & * &  &  &  \\
    & \scriptstyle 1 &  &  &  \\
   \scriptstyle- & \scriptstyle- & \scriptstyle 1 &  &  \\
      \hline
    &  &  & \scriptstyle 1 &  \\
    &  &  & \scriptstyle- & \scriptstyle 1 \\
     \end{pmatrix}&& 
   P_{\ww,\ww s}=
      \begin{pmatrix}[ccc|cc]
   * & * &  &  &  \\
   * & * &  &  &  \\
   * & * & * &  &  \\
   \hline
    &  &  & * &  \\
    &  &  & * & * \\
     \end{pmatrix}
\end{aligned}
\end{equation*}




\end{eg}
\section{typical variety}
\section{stratification and $T$-fixed points}
%%%%%%%%%%%%%%%%%%%%%%%%%%%%%%%%%%%%%%%%%%%%%%%%%%%%%%%%%%%%%%%%%%%%%%%%%%%%%%%%%%%%%%%%%%%%%
\chapter{K-theory and cohomology theory}
%%%%%%%%%%%%%%%%%%%%%%%%%%%%%%%%%%%%%%%%%%%%%%%%%%%%%%%%%%%%%%%%%%%%%%%%%%%%%%%%%%%%%%%%%%%%%
\chapter{Localization theorem}

%%%%%%%%%%%%%%%%%%%%%%%%%%%%%%%%%%%%%%%%%%%%%%%%%%%%%%%%%%%%%%%%%%%%%%%%%%%%%%%%%%%%%%%%%%%%%
\chapter{Excess intersection formula}

%%%%%%%%%%%%%%%%%%%%%%%%%%%%%%%%%%%%%%%%%%%%%%%%%%%%%%%%%%%%%%%%%%%%%%%%%%%%%%%%%%%%%%%%%%%%%
\chapter{From formula to diagram}


%%%%%%%%%%%%%%%%%%%%%%%%%%%%%%%%%%%%%%%%%%%%%%%%%%%%%%%%%%%%%%%%%%%%%%%%%%%%%%%%%%%%%%%%%%%%%
\chapter{Generalization}

%%%%%%%%%%%%%%%%%%%%%%%%%%%%%%%%%%%%%%%%%%%%%%%%%%%%%%%%%%%%%%%%%%%%%%%%%%%%%%%%%%%%%%%%%%%%%
\chapter{Atiyah-Segal completion theorem}


%\bibliographystyle{plain}
%\bibliography{reference}
\end{document}




